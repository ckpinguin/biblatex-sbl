\documentclass[a4paper]{article}

\PassOptionsToPackage{obeyspaces}{url}

\usepackage{microtype}
\usepackage{fontspec}
\usepackage{xparse}
\usepackage{xcolor}
\usepackage{parskip}
\usepackage{titlesec}
\usepackage{listings}
\usepackage{imakeidx}
\makeindex[title=Author Index,intoc,options=-q]
\usepackage[style=sbl,indexing=cite,backend=biber]{biblatex}
\usepackage{hyperref}
\addbibresource{biblatex-sbl.bib}

\DeclareBibliographyCategory{ignore}
\addtocategory{ignore}{pritchard:1969}

\setmonofont{DejaVu Sans Mono}[Scale=MatchLowercase]
\setromanfont{Linux Libertine O}
\setsansfont{Linux Biolinum O}[
  BoldItalicFont={* Bold},
  BoldItalicFeatures={FakeSlant=0.2}
]

\newcommand{\textgreek}[1]{{\greekfont #1}}

\renewcommand{\familydefault}{\sfdefault}

\newfontfamily\greekfont[Script=Greek,Contextuals=Alternate,Ligatures=Required,
  Scale=MatchLowercase]{SBL BibLit}

\renewcommand{\bibfont}{\rm}

\newcommand*{\refbibnamedash}{%
  \leavevmode\raise 0.6ex\hbox to 3em{\hrulefill}.\space}

\titleformat{\paragraph}
{\normalfont\sf\normalsize\bfseries}{\theparagraph}{1em}{}
\titleformat{\subparagraph}
{\normalfont\sf\small\scshape}{\thesubparagraph}{1em}{}
\titlespacing*{\paragraph}{0pt}{2ex plus 1ex minus 0.2ex}{0.5ex}
\titlespacing*{\subparagraph}{0pt}{1ex plus 0.2ex}{0em}

\titleclass{\subsubparagraph}{straight}[\subparagraph]
\titleformat{\subsubparagraph}{\normalfont\small\tt}{}{1em}{}
\titlespacing*{\subsubparagraph}{0pt}{0pt}{0pt}

\setcounter{secnumdepth}{4} % how many sectioning levels to assign numbers to

\setlength{\parskip}{1ex plus 0.5ex minus 0.25ex}

\makeatletter
\patchcmd{\lst@Init}{\par\penalty -50\relax}{\relax}
\makeatother

\hyphenation{nash-ville deu-te-ro-ca-no-ni-cal}
\renewcommand{\slash}{/\penalty\exhyphenpenalty\hspace{0pt}}

\definecolor{biblatex-colour}{rgb}{0.25,0.25,0.65}
\definecolor{reference-colour}{rgb}{0,0.6,0.15}

\ExplSyntaxOn
\NewDocumentCommand \samplemacro { m }
  {
    \subsubparagraph*{#1}
  }
\NewDocumentCommand \sblrefsamplecite { s m m m o o m }
  {
    \IfNoValueTF { #5 }
      {
        \IfNoValueT { #6 }
          {
            \IfBooleanF { #1 }
              {
                \samplemacro{\textbackslash #2\{#7\}}
              }
            \color{biblatex-colour}
            \hspace*{\bibindent}#4\csuse{#3}{#7}
          }
      }
      {
        \IfNoValueTF { #6 }
          {
            \IfBooleanF { #1 }
              {
                \samplemacro{\textbackslash #2[#5]\{#7\}}
              }
            \color{biblatex-colour}
            \hspace*{\bibindent}#4\csuse{#3}[#5]{#7}
          }
          {
            \IfBooleanF { #1 }
              {
                \samplemacro{\textbackslash #2[#5][#6]\{#7\}}
              }
            \color{biblatex-colour}
            \hspace*{\bibindent}#4\csuse{#3}[#5][#6]{#7}
          }
      }
  }
\NewDocumentCommand \samplecite { s m o o m }
  {
    \rmfamily
    \IfBooleanTF { #1 }
      {
        \sblrefsamplecite*{autocite}{cite}{#2.~}[#3][#4]{#5}.\par
      }
      {
        \sblrefsamplecite{autocite}{cite}{#2.~}[#3][#4]{#5}.\par
      }
    \color{black}
  }
\NewDocumentCommand \sampleparencite { s o o m }
  {
    \rmfamily
    {\setlength{\bibindent}{0pt}%
    \IfBooleanTF { #1 }
      {
        \sblrefsamplecite*{parencite}{parencite}{}[#2][#3]{#4}\par
      }
      {
        \sblrefsamplecite{parencite}{parencite}{}[#2][#3]{#4}\par
      }
    }
    \color{black}
  }
\NewDocumentCommand \samplebib { s m }
  {
    \IfBooleanF { #1 }
      {
        \samplemacro{\textbackslash printbibliography}
      }
    \color{biblatex-colour}
    \rmfamily\hangindent\bibindent\bibentrycite{#2}.\par
    \color{black}
  }
\NewDocumentCommand \samplebiblist { s m }
  {
    \IfBooleanF { #1 }
      {
        \samplemacro{\textbackslash printbiblist\{abbreviations\}}
      }
    \color{biblatex-colour}
    \biblistcite{#2}
    \color{black}
  }
\NewDocumentCommand \refbiblist { m m }
  {
    \color{reference-colour}
    \strut\rmfamily\hangindent 6em\rlap{#1}\hskip 6em #2\par
    \color{black}
  }
\ExplSyntaxOff

\newenvironment{biboutput}{%
  \subparagraph{Biblatex Output}
}{\color{black}}

\newenvironment{refimp}{%
  \subparagraph{Reference Implementation}
  \color{reference-colour}
  \rm
}{\par\color{black}}

\lstdefinelanguage{BibTeX}{%
  keywords={%
    @book,@article,@incollection,@suppbook,@mvbook,@collection,@review,@misc,%
    @thesis,@mvreference,@inreference,@mvlexicon,@inlexicon,@unpublished,%
    @commentary,@inbook,@incommentary,@mvcommentary,@seminarpaper,@lexicon,%
    @reference,@mvcollection,@bookinbook,@ancienttext,@classictext,@online,%
    @manual,@conferencepaper,@series%
  },
  emph={%
    author,title,location,publisher,date,shorttitle,translator,edition,preface,%
    related,relatedtype,editor,series,shortseries,number,journaltitle,%
    shortjournal,volume,pages,type,booktitle,bookauthor,origlocation,%
    origpublisher,origdate,pubstate,origlanguage,maintitle,maineditor,part,%
    bookeditor,seriesseries,maintranslator,eprint,eprinttype,doi,url,%
    revdauthor,revdtitle,revdeditor,institution,type,shorthand,xref,%
    note,eprintdate,volumes,shortmaintitle,options,eprintclass,relatedoptions,%
    editortype,crossref,editora,editorb,editorc,editoratype,editorbtyle,%
    editorctype,entrysubtype,sortkey,titleaddon,witheditor,witheditortype,%
    withtranslator,withtranslatortype,eventtitle,venue,eventdate,shortauthor%
  },
  sensitive=false,
  breaklines=true,
  breakatwhitespace=true,
  morecomment=[l][basicstyle]{=\ \{}
}

\lstset{%
 language=BibTeX,
 backgroundcolor=\color{gray!15},
 basicstyle=\ttfamily\small,
 keywordstyle=\color{teal},
 emphstyle=\color{purple}
}

\makeatletter
\lst@InputCatcodes
\def\lst@DefEC{%
 \lst@CCECUse \lst@ProcessLetter
  ^^80^^81^^82^^83^^84^^85^^86^^87^^88^^89^^8a^^8b^^8c^^8d^^8e^^8f%
  ^^90^^91^^92^^93^^94^^95^^96^^97^^98^^99^^9a^^9b^^9c^^9d^^9e^^9f%
  ^^a0^^a1^^a2^^a3^^a4^^a5^^a6^^a7^^a8^^a9^^aa^^ab^^ac^^ad^^ae^^af%
  ^^b0^^b1^^b2^^b3^^b4^^b5^^b6^^b7^^b8^^b9^^ba^^bb^^bc^^bd^^be^^bf%
  ^^c0^^c1^^c2^^c3^^c4^^c5^^c6^^c7^^c8^^c9^^ca^^cb^^cc^^cd^^ce^^cf%
  ^^d0^^d1^^d2^^d3^^d4^^d5^^d6^^d7^^d8^^d9^^da^^db^^dc^^dd^^de^^df%
  ^^e0^^e1^^e2^^e3^^e4^^e5^^e6^^e7^^e8^^e9^^ea^^eb^^ec^^ed^^ee^^ef%
  ^^f0^^f1^^f2^^f3^^f4^^f5^^f6^^f7^^f8^^f9^^fa^^fb^^fc^^fd^^fe^^ff%
  % Greek support
  ^^^^0370^^^^0371^^^^0372^^^^0373^^^^0374^^^^0375^^^^0376^^^^0377%
  ^^^^0378^^^^0379^^^^037a^^^^037b^^^^037c^^^^037d^^^^037e^^^^037f%
  ^^^^0380^^^^0381^^^^0382^^^^0383^^^^0384^^^^0385^^^^0386^^^^0387%
  ^^^^0388^^^^0389^^^^038a^^^^038b^^^^038c^^^^038d^^^^038e^^^^038f%
  ^^^^0390^^^^0391^^^^0392^^^^0393^^^^0394^^^^0395^^^^0396^^^^0397%
  ^^^^0398^^^^0399^^^^039a^^^^039b^^^^039c^^^^039d^^^^039e^^^^039f%
  ^^^^03a0^^^^03a1^^^^03a2^^^^03a3^^^^03a4^^^^03a5^^^^03a6^^^^03a7%
  ^^^^03a8^^^^03a9^^^^03aa^^^^03ab^^^^03ac^^^^03ad^^^^03ae^^^^03af%
  ^^^^03b0^^^^03b1^^^^03b2^^^^03b3^^^^03b4^^^^03b5^^^^03b6^^^^03b7%
  ^^^^03b8^^^^03b9^^^^03ba^^^^03bb^^^^03bc^^^^03bd^^^^03be^^^^03bf%
  ^^^^03c0^^^^03c1^^^^03c2^^^^03c3^^^^03c4^^^^03c5^^^^03c6^^^^03c7%
  ^^^^03c8^^^^03c9^^^^03ca^^^^03cb^^^^03cc^^^^03cd^^^^03ce^^^^03cf%
  ^^^^03d0^^^^03d1^^^^03d2^^^^03d3^^^^03d4^^^^03d5^^^^03d6^^^^03d7%
  ^^^^03d8^^^^03d9^^^^03da^^^^03db^^^^03dc^^^^03dd^^^^03de^^^^03df%
  ^^^^03e0^^^^03e1%
  % Greek extended support
  ^^^^1f00^^^^1f01^^^^1f02^^^^1f03^^^^1f04^^^^1f05^^^^1f06^^^^1f07%
  ^^^^1f08^^^^1f09^^^^1f0a^^^^1f0b^^^^1f0c^^^^1f0d^^^^1f0e^^^^1f0f%
  ^^^^1f10^^^^1f11^^^^1f12^^^^1f13^^^^1f14^^^^1f15^^^^1f16^^^^1f17%
  ^^^^1f18^^^^1f19^^^^1f1a^^^^1f1b^^^^1f1c^^^^1f1d^^^^1f1e^^^^1f1f%
  ^^^^1f20^^^^1f21^^^^1f22^^^^1f23^^^^1f24^^^^1f25^^^^1f26^^^^1f27%
  ^^^^1f28^^^^1f29^^^^1f2a^^^^1f2b^^^^1f2c^^^^1f2d^^^^1f2e^^^^1f2f%
  ^^^^1f30^^^^1f31^^^^1f32^^^^1f33^^^^1f34^^^^1f35^^^^1f36^^^^1f37%
  ^^^^1f38^^^^1f39^^^^1f3a^^^^1f3b^^^^1f3c^^^^1f3d^^^^1f3e^^^^1f3f%
  ^^^^1f40^^^^1f41^^^^1f42^^^^1f43^^^^1f44^^^^1f45^^^^1f46^^^^1f47%
  ^^^^1f48^^^^1f49^^^^1f4a^^^^1f4b^^^^1f4c^^^^1f4d^^^^1f4e^^^^1f4f%
  ^^^^1f50^^^^1f51^^^^1f52^^^^1f53^^^^1f54^^^^1f55^^^^1f56^^^^1f57%
  ^^^^1f58^^^^1f59^^^^1f5a^^^^1f5b^^^^1f5c^^^^1f5d^^^^1f5e^^^^1f5f%
  ^^^^1f60^^^^1f61^^^^1f62^^^^1f63^^^^1f64^^^^1f65^^^^1f66^^^^1f67%
  ^^^^1f68^^^^1f69^^^^1f6a^^^^1f6b^^^^1f6c^^^^1f6d^^^^1f6e^^^^1f6f%
  ^^^^1f70^^^^1f71^^^^1f72^^^^1f73^^^^1f74^^^^1f75^^^^1f76^^^^1f77%
  ^^^^1f78^^^^1f79^^^^1f7a^^^^1f7b^^^^1f7c^^^^1f7d^^^^1f7e^^^^1f7f%
  ^^^^1f80^^^^1f81^^^^1f82^^^^1f83^^^^1f84^^^^1f85^^^^1f86^^^^1f87%
  ^^^^1f88^^^^1f89^^^^1f8a^^^^1f8b^^^^1f8c^^^^1f8d^^^^1f8e^^^^1f8f%
  ^^^^1f90^^^^1f91^^^^1f92^^^^1f93^^^^1f94^^^^1f95^^^^1f96^^^^1f97%
  ^^^^1f98^^^^1f99^^^^1f9a^^^^1f9b^^^^1f9c^^^^1f9d^^^^1f9e^^^^1f9f%
  ^^^^1fa0^^^^1fa1^^^^1fa2^^^^1fa3^^^^1fa4^^^^1fa5^^^^1fa6^^^^1fa7%
  ^^^^1fa8^^^^1fa9^^^^1faa^^^^1fab^^^^1fac^^^^1fad^^^^1fae^^^^1faf%
  ^^^^1fb0^^^^1fb1^^^^1fb2^^^^1fb3^^^^1fb4^^^^1fb5^^^^1fb6^^^^1fb7%
  ^^^^1fb8^^^^1fb9^^^^1fba^^^^1fbb^^^^1fbc^^^^1fbd^^^^1fbe^^^^1fbf%
  ^^^^1fc0^^^^1fc1^^^^1fc2^^^^1fc3^^^^1fc4^^^^1fc5^^^^1fc6^^^^1fc7%
  ^^^^1fc8^^^^1fc9^^^^1fca^^^^1fcb^^^^1fcc^^^^1fcd^^^^1fce^^^^1fcf%
  ^^^^1fd0^^^^1fd1^^^^1fd2^^^^1fd3^^^^1fd4^^^^1fd5^^^^1fd6^^^^1fd7%
  ^^^^1fd8^^^^1fd9^^^^1fda^^^^1fdb^^^^1fdc^^^^1fdd^^^^1fde^^^^1fdf%
  ^^^^1fe0^^^^1fe1^^^^1fe2^^^^1fe3^^^^1fe4^^^^1fe5^^^^1fe6^^^^1fe7%
  ^^^^1fe8^^^^1fe9^^^^1fea^^^^1feb^^^^1fec^^^^1fed^^^^1fee^^^^1fef%
  ^^^^1ff0^^^^1ff1^^^^1ff2^^^^1ff3^^^^1ff4^^^^1ff5^^^^1ff6^^^^1ff7%
  ^^^^1ff8^^^^1ff9^^^^1ffa^^^^1ffb^^^^1ffc^^^^1ffd^^^^1ffe^^^^1fff%
  % General punctuation
  ^^^^201c^^^^201d%
  ^^00}
\lst@RestoreCatcodes
\makeatother

\title{Biblatex-SBL Examples and Test File}
\author{David Purton}
\date{}

\begin{document}

\maketitle

\tableofcontents

\clearpage

\setcounter{section}{5}

\section{Notes and Bibliographies}

\setcounter{subsection}{1}
\subsection{General Examples: Books}

\subsubsection{A Book by a Single Author}

\begin{lstlisting}
@book{talbert:1992,
  author = {Talbert, Charles H.},
  title = {Reading John: A Literary and Theological Commentary on the Fourth Gospel and the Johannine Epistles},
  location = {New York},
  publisher = {Crossroad},
  date = {1992}
}
\end{lstlisting}

\begin{biboutput}
  \samplecite{15}[127]{talbert:1992}
  \samplecite{19}[22]{talbert:1992}
  \samplebib{talbert:1992}
\end{biboutput}

\begin{refimp}
  \hspace*{\bibindent}15. Charles H. Talbert, \emph{Reading John: A Literary
  and Theological Commentary on the Fourth Gospel and the Johannine Epistles}
  (New York: Crossroad, 1992), 127.

  \hspace*{\bibindent}19. Talbert, \emph{Reading John,} 22.

  \hangindent\bibindent Talbert, Charles H. \emph{Reading John: A Literary and
  Theological Commentary on the Fourth Gospel and the Johannine Epistles.} New
  York: Crossroad, 1992.
\end{refimp}

\subsubsection{A Book by Two or Three Authors}

\begin{lstlisting}
@book{robinson+koester:1971,
  author = {Robinson, James M. and Koester, Helmut},
  title = {Trajectories through Early Christianity},
  location = {Philadelphia},
  publisher = {Fortress},
  date = {1971}
}
\end{lstlisting}  

\begin{biboutput}
  \samplecite{4}[237]{robinson+koester:1971}
  \samplecite{12}[23]{robinson+koester:1971}
  \samplebib{robinson+koester:1971}
\end{biboutput}

\begin{refimp}
  \hspace*{\bibindent}4. James M. Robinson and Helmut Koester,
  \emph{Trajectories through Early Christianity} (Philadelphia: Fortress,
  1971), 237.

  \hspace*{\bibindent}12. Robinson and Koester, \emph{Trajectories through
  Early Christianity,} 23.

  \hangindent\bibindent Robinson, James M., and Helmut Koester. \emph{Trajectories
  through Early Christianity.} Philadelphia: Fortress, 1971.
\end{refimp}

\subsubsection{A Book by More than Three Authors}

\begin{lstlisting}
@book{scott+etal:1993,
  author = {Scott, Bernard Brandon and Dean, Margaret and Sparks, Kristen and LaZar, Frances},
  title = {Reading New Testament Greek},
  location = {Peabody, MA},
  publisher = {Hendrickson},
  date = {1993}
}
\end{lstlisting}  

\begin{biboutput}
  \samplecite{7}[53]{scott+etal:1993}
  \samplecite{9}[42]{scott+etal:1993}
  \samplebib{scott+etal:1993}
\end{biboutput}

\begin{refimp}
  \hspace*{\bibindent}7. Bernard Brandon Scott et al., \emph{Reading New
  Testament Greek} (Peabody, MA: Hendrickson, 1993), 53.
  
  \hspace*{\bibindent}9. Scott et al., \emph{Reading New Testament Greek,} 42.

  \hangindent\bibindent Scott, Bernard Brandon, Margaret Dean, Kristen Sparks,
  and Frances LaZar. \emph{Reading New Testament Greek.} Peabody, MA:
  Hendrickson, 1993.
\end{refimp}

\subsubsection{A Translated Volume}

\begin{lstlisting}
@book{egger:1996,
  author = {Egger, Wilhelm},
  title = {How to Read the New Testament: An Introduction to Linguistic and Historical-Critical Methodology},
  shorttitle = {How to Read},
  translator = {Heinegg, Peter},
  location = {Peabody, MA},
  publisher = {Hendrickson},
  date = {1996}
}
\end{lstlisting}  

\begin{biboutput}
  \samplecite{14}[28]{egger:1996}
  \samplecite{18}[291]{egger:1996}
  \samplebib{egger:1996}
\end{biboutput}

\begin{refimp}
  \hspace*{\bibindent}14. Wilhelm Egger, \emph{How to Read the New Testament:
  An Introduction to Linguistic and Historical-Critical Methodology,} trans.\@
  Peter Heinegg (Peabody, MA: Hendrickson, 1996), 28.

  \hspace*{\bibindent}18. Egger, \emph{How to Read,} 291.
  
  \hangindent\bibindent Egger, Wilhelm. \emph{How to Read the New Testament:
  An Introduction to Linguistic and Historical-Critical Methodology.}
  Translated by Peter Heinegg. Peabody, MA: Hendrickson, 1996.
\end{refimp}

\subsubsection{The Full History of a Translated Volume}

\begin{lstlisting}
@book{wellhausen:1883,
  author = {Wellhausen, Julius},
  title = {Prolegomena zur Geschichte Israels},
  edition = {2},
  location = {Berlin},
  publisher = {Reimer},
  date = {1883}
}

@book{wellhausen:1885,
  author = {Wellhausen, Julius},
  title = {Prolegomena to the History of Israel},
  translator = {Black, J. Sutherland and Enzies, A.},
  withtranslator = {Smith, W. Robertson},
  withtranslatortype = {withpreface},
  location = {Edinburgh},
  publisher = {Black},
  related = {wellhausen:1883},
  relatedtype = {translationof},
  date = {1885}
}

@book{wellhausen:1957,
  author = {Wellhausen, Julius},
  title = {Prolegomena to the History of Ancient Israel},
  location = {New York},
  publisher = {Meridian Books},
  related = {wellhausen:1885},
  relatedtype = {reprintof},
  date = {1957}
}
\end{lstlisting}

\begin{biboutput}
  \samplemacro{\textbackslash autocite[296]\{wellhausen:1957\}}
  \sloppy\samplecite*{3}[296]{wellhausen:1957}
  \samplebib{wellhausen:1957}
\end{biboutput}

\begin{refimp}
  \hspace*{\bibindent}3. Julius Wellhausen, \emph{Prolegomena to the History
  of Ancient Israel} (New York: Meridian Books, 1957), 296; repr. of
  \emph{Prolegomena to the History of Israel,} trans.\@ J.~Sutherland Black and
  A.~Enzies, with preface by W.~Robertson Smith (Edinburgh: Black, 1885);
  trans.\@ of \emph{Prolegomena zur Geschichte Israels,} 2nd~ed. (Berlin:
  Reimer, 1883).

  \hangindent\bibindent Julius Wellhausen,\footnote{Should be “Wellhausen,
  Julius.”?} \emph{Prolegomena to the History of Ancient Israel.} New York:
  Meridian Books, 1957. Reprint of \emph{Prolegomena to the History of
  Israel.} Translated by J.~Sutherland Black and A.~Enzies, with preface by
  W.~Robertson Smith. Edinburgh: Black, 1885. Translation of \emph{Prolegomena
  zur Geschichte Israels.} 2nd~ed. Berlin: Reimer, 1883.
\end{refimp}

\subsubsection{A Book with One Editor}

\begin{lstlisting}
@book{tigay:1985,
  editor = {Tigay, Jeffrey H.},
  title = {Empirical Models for Biblical Criticism},
  shorttitle = {Empiracle Models},
  location = {Philadelphia},
  publisher = {University of Pennsylvania Press},
  date = {1985}
}
\end{lstlisting}  

\begin{biboutput}
  \samplecite{5}[35]{tigay:1985}
  \samplecite{9}[38]{tigay:1985}
  \samplebib{tigay:1985}
\end{biboutput}

\begin{refimp}
  \hspace*{\bibindent}5. Jeffrey H. Tigay, ed., \emph{Empirical Models for
  Biblical Criticism} (Philadelphia: University of Pennsylvania Press, 1985),
  35.

  \hspace*{\bibindent}9. Tigay, \emph{Empirical Models,} 38.

  \hangindent\bibindent Tigay, Jeffrey H., ed. \emph{Empirical Models for
  Biblical Criticism.} Philadelphia: University of Pennsylvania Press,
  1985.
\end{refimp}

\subsubsection{A Book with Two or Three Editors}

\begin{lstlisting}
@book{kaltner+mckenzie:2002,
  editor = {Kaltner, John and McKenzie, Steven L.},
  title = {Beyond Babel: A Handbook for Biblical Hebrew and Related Languages},
  series = {Resources for Biblical Study},
  shortseries = {RBS},
  number = {42},
  location = {Atlanta},
  publisher = {Society of Biblical Literature},
  date = {2002}
}
\end{lstlisting}  

\begin{biboutput}
  \samplecite{44}[xii]{kaltner+mckenzie:2002}
  \samplecite{47}[viii]{kaltner+mckenzie:2002}
  \samplebib{kaltner+mckenzie:2002}
  \samplebiblist{kaltner+mckenzie:2002}
\end{biboutput}

\begin{refimp}
  \hspace*{\bibindent}44. John Kaltner and Steven L. McKenzie, eds.,
  \emph{Beyond Babel: A Handbook for Biblical Hebrew and Related Languages,}
  RBS~42 (Atlanta: Society of Biblical Literature, 2002), xii.

  \hspace*{\bibindent}47. Kaltner and McKenzie,\footnote{Should include
  \emph{“Beyond Babel,”}?} viii.

  \hangindent\bibindent Kaltner, John, and Steven L. McKenzie, eds.
  \emph{Beyond Babel: A Handbook for Biblical Hebrew and Related Languages.}
  RBS~42. Atlanta: Society of Biblical Literature, 2002.

  \refbiblist{RBS}{Resources for Biblical Study}
\end{refimp}

\subsubsection{A Book with Four or More Editors}

\begin{lstlisting}
@book{oates+etal:2001,
  editor = {Oates, John F. and Willis, William H. and Bagnall, Roger S. and Worp, Klass A.},
  title = {Checklist of Editions of Greek and Latin Papyri, Ostraca, and Tablets},
  edition = {5},
  series = {Bulletin of the American Society of Papyrologists, Supplements},
  shortseries = {BASPSup},
  number = {9},
  location = {Oakville, CT},
  publisher = {American Society of Papyrologists},
  date = {2001}
}
\end{lstlisting}  

\begin{biboutput}
  \samplecite{4}[10]{oates+etal:2001}
  \samplebib{oates+etal:2001}
  \samplebiblist{oates+etal:2001}
\end{biboutput}

\begin{refimp}
  \hspace*{\bibindent}4. John F. Oates et al., eds., \emph{Checklist of
  Editions of Greek and Latin Papyri, Ostraca, and Tablets,} 5th~ed., BASPSup
  9 (Oakville, CT: American Society of Papyrologists, 2001), 10.

  \hangindent\bibindent Oates, John F., William H. Willis, Roger S. Bagnall,
  and Klaas A. Worp, eds. \emph{Checklist of Editions of Greek and Latin
  Papyri, Ostraca, and Tablets.} 5th~ed. BASPSup~9. Oakville, CT: American
  Society of Papyrologists, 2001.

  \refbiblist{BASPSup}{Bulletin of the American Society of Papyrologists,
  Supplements}
\end{refimp}

\subsubsection{A Book with Both Author and Editor}

\begin{lstlisting}
@book{schillebeeckx:1986,
  author = {Schillebeeckx, Edward},
  title = {The Schillebeeckx Reader},
  editor = {Schreiter, Robert J.},
  location = {Edinburgh},
  publisher = {T\&T Clark},
  date = {1986}
}
\end{lstlisting}  

\begin{biboutput}
  \samplecite{45}[20]{schillebeeckx:1986}
  \samplebib{schillebeeckx:1986}
\end{biboutput}

\begin{refimp}
  \hspace*{\bibindent}45. Edward Schillebeeckx, \emph{The Schillebeeckx
  Reader,} ed.\@ Robert J. Schreiter (Edinburgh: T\&T Clark, 1986), 20.

  \hangindent\bibindent Schillebeeckx, Edward. \emph{The Schillebeeckx
  Reader.} Edited by Robert J. Schreiter. Edinburgh: T\&T Clark, 1986.
\end{refimp}

\subsubsection{A Book with Author, Editor, and Translator}

\begin{lstlisting}
@book{blass+debrunner:1982,
  author = {Blass, Friedrich and Debrunner, Albert},
  title = {Grammatica del greco del Nuovo Testamento},
  editor = {Rehkopf, Friedrich},
  translator = {Pisi, Giordana},
  location = {Brescia},
  publisher = {Paideia},
  date = {1982}
}
\end{lstlisting}  

\begin{biboutput}
  \samplecite{3}[40]{blass+debrunner:1982}
  \samplebib{blass+debrunner:1982}
\end{biboutput}

\begin{refimp}
  \hspace*{\bibindent}3. Friedrich Blass and Albert Debrunner,
  \emph{Grammatica del greco del Nuovo Testamento,} ed.\@ Friedrich Rehkopf,
  trans.\@ Giordana Pisi (Brescia: Paideia, 1982), 40.

  \hangindent\bibindent Blass, Friedrich, and Albert Debrunner.
  \emph{Grammatica del greco del Nuovo Testamento.} Edited by Friedrich
  Rehkopf. Translated by Giordana Pisi. Brescia: Paideia, 1982.
\end{refimp}

\subsubsection{A Title in a Modern Work Citing a Nonroman Alphabet}

\begin{lstlisting}
@article{irvine:2014,
  author = {Irvine, Stuart A.},
  title = {Idols \mkbibbrackets{\mkbibemph{ktbwnm}}: A note on Hosea 13:2a},
  journaltitle = {Journal of Biblical Literature},
  shortjournal = {JBL},
  volume = {133},
  date = {2014},
  pages = {509-517}
}
\end{lstlisting}

\begin{biboutput}
  \samplecite{34}{irvine:2014}
  \samplebiblist{irvine:2014}
\end{biboutput}

\begin{refimp}
  \hspace*{\bibindent}34. Stuart A. Irvine, “Idols [\emph{ktbwnm}]: A Note on
  Hosea 13:2a,” \emph{JBL}~133 (2014): 509–17.

  \refbiblist{\emph{JBL}}{\emph{Journal of Biblical Literature}}
\end{refimp}

\subsubsection{An Article in and Edited Volume}

\begin{lstlisting}
@incollection{attridge:1986,
  author = {Attridge, Harold W.},
  title = {Jewish Historiography},
  pages = {311-343},
  booktitle = {Early Judaism and Its Modern Interpreters},
  editor = {Kraft, Robert A. and Nickelsburg, George W. E.},
  location = {Philadelphia and Atlanta},
  publisher = {Fortress and Scholars Press},
  date = {1986}
}
\end{lstlisting}  

\begin{biboutput}
  \samplecite{3}{attridge:1986}
  \samplecite{6}{attridge:1986}
  \samplebib{attridge:1986}
\end{biboutput}

\begin{refimp}
  \hspace*{\bibindent}3. Harold W. Attridge, “Jewish Historiography,” in
  \emph{Early Judaism and Its Modern Interpreters,} ed.\@ Robert A. Kraft and
  George W. E. Nickelsburg (Philadelphia: Fortress; Atlanta: Scholars Press,
  1986), 311–43.
  
  \hspace*{\bibindent}6. Attridge, “Jewish Historiography,” 314–17.

  Attridge, Harold A. “Jewish Historiography.” Pages 311–43 in \emph{Early
  Judaism and Its Modern Interpreters.} Edited by Robert A. Kraft and George
  W. E. Nickelsburg. Philadelphia: Fortress; Atlanta: Scholars Press, 1986.
\end{refimp}

\subsubsection{An Article in a Festschrift}

\begin{lstlisting}
@incollection{vanseters:1995,
  author = {Van Seters, John},
  title = {The Theology of the Yahwist: A Preliminary Sketch},
  shorttitle = {Theology of the Yahwist},
  pages = {219-228},
  booktitle = {“Wer ist wie du, Herr, unter den Göttern?”: Studien zur Theologie und Religionsgeschichte Israels für Otto Kaiser zum~70. Geburtstag},
  editor = {Kottsieper, Ingo and others},
  location = {Göttingen},
  publisher = {Vandenhoeck \& Ruprecht},
  date = {1995}
}
\end{lstlisting}  

\begin{biboutput}
  \samplecite{8}{vanseters:1995}
  \samplecite{17}[222]{vanseters:1995}
  \samplebib{vanseters:1995}
\end{biboutput}

\begin{refimp}
  \hspace*{\bibindent}8. John Van Seters, “The Theology of the Yahwist: A
  Preliminary Sketch,” in \emph{“Wer ist wie du, Herr, unter den Göttern?”:
  Studien zur Theologie und Religionsgeschichte Israels für Otto Kaiser zum
70. Geburtstag,} ed.\@ Ingo Kottsieper et al. (Göttingen: Vandenhoeck \&
  Ruprecht, 1995), 219–28.

  \hspace*{\bibindent}17. Van Seters, “Theology of the Yahwist,” 222.

  \hangindent\bibindent Van Seters, John. “The Theology of the Yahwist: A
  Preliminary Sketch.” Pages~219–28 in \emph{“Wer ist wie du, Herr, unter den
  Göttern?”: Studien zur Theologie und Religionsgeschichte Israels für Otto
  Kaiser zum~70. Geburtstag.} Edited by Ingo Kottsieper et al. Göttingen:
  Vandenhoeck \& Ruprecht, 1995.
\end{refimp}

\subsubsection{An Introduction, Preface, or Foreword Written by Someone Other
Than the Author}

\begin{lstlisting}
@suppbook{boers:1996,
  author = {Boers, Hendrikus},
  type = {introduction},
  booktitle = {How to Read the New Testament: An Introduction to Linguistic and Historical-Critical Methodology},
  bookauthor = {Egger, Wilhelm},
  translator = {Heinegg, Peter},
  location = {Peabody, MA},
  publisher = {Hendrickson},
  date = {1996}
}
\end{lstlisting}  

\begin{biboutput}
  \samplecite{2}[xi-xii]{boers:1996}
  \samplecite{6}[xi-xx]{boers:1996}
  \samplebib{boers:1996}
\end{biboutput}

\begin{refimp}
  \hspace*{\bibindent}2. Hendrikus Boers, introduction to \emph{How to Read
  the New Testament: An Introduction to Linguistic and Historical-Critical
  Methodology,} by Wilhelm Egger, trans.\@ Peter Heinegg (Peabody; MA:
  Hendrickson, 1996), xi–xxi.

  \hspace*{\bibindent}6. Boers, introduction, xi–xx.

  \hangindent\bibindent Boers, Hendrikus. Introduction to \emph{How to Read
  the New Testament: An Introduction to Linguistic and Historical-Critical
  Methodology,} by Wilhelm Egger. Translated by Peter Heinegg. Peabody; MA:
  Hendrickson, 1996.
\end{refimp}

\subsubsection{Multiple Publishers for a Single Book}

\begin{lstlisting}
@book{gerhardsson:1961,
  author = {Gerhardsson, Birger},
  title = {Memory and Manuscript: Oral Tradition and Written Transmission in Rabbinic Judaism and Early Christianity},
  series = {Acta Seminarii Neotestamentici Upsaliensis},
  shortseries = {ASNU},
  number = {22},
  location = {Lund and Copenhagen},
  publisher = {Gleerup and Munksgaard},
  date = {1961}
}
\end{lstlisting}

\begin{biboutput}
  \samplebib{gerhardsson:1961}
  \samplebiblist{gerhardsson:1961}
\end{biboutput}

\begin{refimp}
  \hangindent\bibindent Birger Gerhardsson,\footnote{Should be “Gerhardsson,
  Birger.”?} \emph{Memory and Manuscript: Oral Tradition and Written
  Transmission in Rabbinic Judaism and Early Christianity.} ASNU~22. Lund:
  Gleerup; Copenhagen: Munksgaard, 1961.

  \refbiblist{ASNU}{Acta Seminarii Neotestamentici Upsaliensis}
\end{refimp}

\subsubsection{A Revised Edition}

\begin{lstlisting}
@book{pritchard:1969,
  editor = {Pritchard, James B.},
  title = {Ancient Near Eastern Texts Relating to the Old Testament},
  edition = {3},
  location = {Princeton},
  publisher = {Princeton University Press},
  date = {1969}
}
\end{lstlisting}  

\begin{biboutput}
  \samplecite{87}[xxi]{pritchard:1969}
  \samplebib{pritchard:1969}
\end{biboutput}

\begin{refimp}
  \hspace*{\bibindent}87. James B. Pritchard, ed., \emph{Ancient Near Eastern
  Texts Relating to the Old Testament,} 3rd~ed. (Princeton: Princeton
  University Press, 1969), xxi.

  \hangindent\bibindent Pritchard, James B., ed. \emph{Ancient Near Eastern
  Texts Relating to the Old Testament.} 3rd~ed. Princeton: Princeton
  University Press, 1969.
\end{refimp}

\begin{lstlisting}
@book{blenkinsopp:1996,
  author = {Blenkinsopp, Joseph},
  title = {A History of Prophecy in Israel},
  edition = {rev.\@ and enl.\@ ed.},
  location = {Louisville},
  publisher = {Westminster John Knox},
  date = {1996}
}
\end{lstlisting}  

\begin{biboutput}
  \samplecite{56}[81]{blenkinsopp:1996}
  \samplebib{blenkinsopp:1996}
\end{biboutput}

\begin{refimp}
  \hspace*{\bibindent}56. Joseph Blenkinsopp, \emph{A History of Prophecy in
  Israel,} rev.\@ and enl.\@ ed. (Louisville: Westminster John Knox, 1996), 81.

  \hangindent\bibindent Blenkinsopp, Joseph. \emph{A History of Prophecy in
  Israel.} Rev.\@ and enl.\@ ed. Louisville: Westminster John Knox, 1996.
\end{refimp}

\subsubsection{A Recent Reprint Title}

\begin{lstlisting}
@book{vanseters:1997,
  author = {Van Seters, John},
  title = {In Search of History: Histeriography in the Ancient World and the Origins of Biblical History},
  origlocation = {New Haven},
  origpublisher = {Yale University Press},
  origdate = {1983},
  location = {Winona Lake, IN},
  publisher = {Eisenbrauns},
  date = {1997}
}
\end{lstlisting}  

\begin{biboutput}
  \samplecite{5}[35]{vanseters:1997}
  \samplebib{vanseters:1997}
\end{biboutput}

\begin{refimp}
  \hspace*{\bibindent}5. John Van Seters, \emph{In Search of History:
  Historiography in the Ancient World and the Origins of Biblical History}
  (New Haven: Yale University Press, 1983; repr., Winona Lake, IN:
  Eisenbrauns, 1997), 35.

  \hangindent\bibindent Van Seters, John. \emph{In Search of History:
  Historiography in the Ancient World and the Origins of Biblical History.}
  New Haven: Yale University Press, 1983. Repr., Winona Lake, IN: Eisenbrauns,
  1997.
\end{refimp}

\subsubsection{A Reprint Title in the Public Domain}

\begin{lstlisting}
@book{deissmann:1995,
  author = {Deissmann, Gustav Adolf},
  title = {Light from the Ancient East: The New Testament Illustrated by Recently Discovered Texts of the Graeco-Roman World},
  translator = {Strachan, Lionel R. M.},
  origdate = {1927},
  location = {Peabody, MA},
  publisher = {Hendrickson},
  date = {1995}
}
\end{lstlisting}  

\begin{biboutput}
  \samplecite{5}[55]{deissmann:1995}
  \samplebib{deissmann:1995}
\end{biboutput}

\begin{refimp}
  \hspace*{\bibindent}5. Gustav Adolf Deissmann, \emph{Light from the Ancient
  East: The New Testament Illustrated by Recently Discovered Texts of the
  Graeco-Roman World} (trans.\@ Lionel R. M. Strachan;\footnote{Should be “,
  trans.\@ Lionel R.\ M.\ Strachan (”?} 1927; repr., Peabody, MA: Hendrickson,
  1995), 55.

  \hangindent\bibindent Deissmann, Gustav Adolf. \emph{Light from the Ancient
  East: The New Testament Illustrated by Recently Discovered Texts of the
  Graeco-Roman World.} Translated by Lionel R. M. Strachan. 1927. Repr., Peabody,
  MA: Hendrickson, 1995.
\end{refimp}

\subsubsection{A Forthcoming Book}

\begin{lstlisting}
@book{harrison+welborn:forthcoming,
  editor = {Harrison, James R. and Welborn, L. L.},
  title = {The First Urban Churches 2: Roman Corinth},
  shorttitle = {Roman Corinth},
  series = {Writings from the Greco-Roman World Supplement Series},
  shortseries = {WGRWSup},
  location = {Atlanta},
  publisher = {SBL Press},
  pubstate = {forthcoming}
}
\end{lstlisting}

\begin{biboutput}
  \samplecite{9}{harrison+welborn:forthcoming}
  \samplecite{12}[201]{harrison+welborn:forthcoming}
  \samplebib{harrison+welborn:forthcoming}
  \samplebiblist{harrison+welborn:forthcoming}
\end{biboutput}

\begin{refimp}
  \hspace*{\bibindent}9. James R. Harrison and L. L. Welborn, eds., \emph{The
  First Urban Churches 2: Roman Corinth,} WGRWSup (Atlanta: SBL Press,
  forthcoming).

  \hspace*{\bibindent}12. Harrison and Welborn, \emph{Roman Corinth,} 201.

  \hangindent\bibindent Harrison, James R.\footnote{Should be “Harrison, James
  R.,”} and L. L. Welborn, eds. \emph{The First Urban Churches 2: Roman
  Corinth.} WGRWSup. Atlanta: SBL Press, forthcoming.

  \refbiblist{WGRWSup}{Writings from the Greco-Roman World Supplement Series}
\end{refimp}

\subsubsection{A Multivolume Work}

\begin{lstlisting}
@mvbook{harnack:1896-1905,
  author = {Harnack, Adolf},
  title = {History of Dogma},
  translator = {Buchanan, Neil},
  origlanguage = {the 3rd German ed.},
  volumes = {7},
  location = {Boston},
  publisher = {Little, Brown},
  date = {1896/1905}
}
\end{lstlisting}  

\begin{biboutput}
  \samplecite{5}{harnack:1896-1905}
  \samplecite{9}[2:126]{harnack:1896-1905}
  \samplebib{harnack:1896-1905}
\end{biboutput}

\begin{refimp}
  \hspace*{\bibindent}5. Adolf Harnack, \emph{History of Dogma,} trans.\@ Neil
  Buchanan, 7 vols. (Boston: Little, Brown).

  \hspace*{\bibindent}9. Harnack, \emph{History of Dogma,} 2:126.

  \hangindent\bibindent Harnack, Adolf. \emph{History of Dogma.} Translated
  from the 3rd German ed.\@ by Neil Buchanan. 7 vols. Boston: Little, Brown,
  1896–1905.
\end{refimp}

\subsubsection{A Titled Volume in a Multivolume, Edited Work}

\begin{lstlisting}
@collection{winter+clarke:1993,
  editor = {Winter, Bruce W. and Clarke, Andrew D.},
  title = {The Book of Acts in Its Ancient Literary Setting},
  shorttitle = {Book of Acts},
  volume = {1},
  maintitle = {The Book of Acts in Its First Century Setting},
  maineditor = {Winter, Bruce W.},
  location = {Grand Rapids},
  publisher = {Eerdmans},
  date = {1993}
}
\end{lstlisting}  

\begin{biboutput}
  \samplecite{5}[25]{winter+clarke:1993}
  \samplecite{16}[25]{winter+clarke:1993}
  \samplebib{winter+clarke:1993}
\end{biboutput}

\begin{refimp}
  \hspace*{\bibindent}5. Bruce W. Winter and Andrew D. Clarke, eds., \emph{The
  Book of Acts in Its Ancient Literary Setting,} vol.~1 of \emph{The Book of
  Acts in Its First Century Setting,} ed.\@ Bruce W. Winter (Grand Rapids:
  Eerdmans, 1993), 25.

  \hspace*{\bibindent}16. Winter and Clarke, \emph{Book of Acts,} 25.

  \hangindent\bibindent Winter, Bruce W., and Andrew D. Clarke, eds. \emph{The
  Book of Acts in Its Ancient Literary Setting.} Vol.~1 of \emph{The Book of
  Acts in Its First Century Setting.} Edited by Bruce W. Winter. Grand Rapids:
  Eerdmans, 1993.
\end{refimp}

\subsubsection{A Chapter within a Multivolume Work}

\begin{lstlisting}
@incollection{mason:1996,
  author = {Mason, Steve},
  title = {Josephus on Canon and Scriptures},
  pages = {217-235},
  volume = {1},
  part = {1},
  maintitle = {Hebrew Bible\slash Old Testament: The History of Its Interpretation},
  editor = {Sæbø, Magne},
  location = {Göttingen},
  publisher = {Vandenhoeck \& Ruprecht},
  date = {1996}
}
\end{lstlisting}

\begin{biboutput}
  \samplecite{24}{mason:1996}
  \samplecite{28}[224]{mason:1996}
  \samplebib{mason:1996}
\end{biboutput}

\begin{refimp}
  \hspace*{\bibindent}24. Steve Mason, “Josephus on Canon and Scriptures,” in
  \emph{Hebrew Bible\slash Old Testament: The History of Its Interpretation,}
  ed.\@ Magne Saebø (Göttingen: Vandenhoeck \& Ruprecht, 1996),
  1.1:217–335.\footnote{Should be “1.1:217–35.”?}

  \hspace*{\bibindent}28. Mason, “Josephus on Canon and Scriptures,” 224.

  \hangindent\bibindent Mason, Steve. “Josephus on Canon and Scriptures.”
  Pages 217–35 in vol.~1, part 1 of \emph{Hebrew Bible\slash Old Testament:
  The History of Its Interpretation.} Edited by Magne Saebø. Göttingen:
  Vandenhoeck \& Ruprecht, 1996.
\end{refimp}

\subsubsection{A Chapter within a Titled Volume in a Multivolume Work}

\begin{lstlisting}
@incollection{peterson:1993,
  author = {Peterson, David},
  title = {The Motif of Fulfilment and the Purpose of Luke-Acts},
  shorttitle = {Motif of Fulfilment},
  pages = {83-104},
  booktitle = {The Book of Acts in Its Ancient Literary Setting},
  bookeditor = {Winter, Bruce W. and Clarke, Andrew D.},
  volume = {1},
  maintitle = {The Book of Acts in Its First Century Setting},
  editor = {Winter, Bruce W.},
  location = {Grand Rapids},
  publisher = {Eerdmans},
  date = {1993}
}
\end{lstlisting}  

\begin{biboutput}
  \samplecite{66}{peterson:1993}
  \samplecite{78}[92]{peterson:1993}
  \samplebib{peterson:1993}
\end{biboutput}

\begin{refimp}
  \hspace*{\bibindent}66. David Peterson, “The Motif of Fulfilment and the
  Purpose of Luke-Acts,” in \emph{The Book of Acts in Its Ancient Literary
  Setting,} ed.\@ Bruce W. Winter and Andrew D. Clarke, vol.~1 of \emph{The
  Book of Acts in Its First Century Setting,} ed.\@ Bruce W. Winter (Grand
  Rapids: Eerdmans, 1993), 83–104.

  \hspace*{\bibindent}78. Peterson, “Motif of Fulfilment,” 92.

  \hangindent\bibindent David Peterson,\footnote{Should be “Peterson,
  David.”?} “The Motif of Fulfilment and the Purpose of Luke-Acts.”
  Pages~83–104 in \emph{The Book of Acts in Its Ancient Literary Setting.}
  Edited by Bruce W. Winter and Andrew D. Clarke. Vol.~1 of \emph{The Book of
  Acts in Its First Century Setting.} Edited by Bruce W. Winter. Grand Rapids:
  Eerdmans, 1993.
\end{refimp}

\subsubsection{A Work in a Series}

\begin{lstlisting}
@book{hofius:1989,
  author = {Hofius, Otfried},
  title = {Paulusstudien},
  series = {Wissenschaftliche Untersuchungen zum Neuen Testament},
  shortseries = {WUNT},
  number = {51},
  location = {Tübingen},
  publisher = {Mohr Siebeck},
  date = {1989}
}
\end{lstlisting}  

\begin{biboutput}
  \samplecite{12}[122]{hofius:1989}
  \samplecite{14}[124]{hofius:1989}
  \samplebib{hofius:1989}
  \samplebiblist{hofius:1989}
\end{biboutput}

\begin{refimp}
  \hspace*{\bibindent}12. Otfried Hofius, \emph{Paulusstudien,} WUNT 51
  (Tübingen: Mohr Siebeck, 1989), 122.

  \hspace*{\bibindent}14. Hofius, \emph{Paulusstudien,} 124.

  Hofius, Otfried. \emph{Paulusstudien.} WUNT~51. Tübingen: Mohr Siebeck,
  1989.
  
  \refbiblist{WUNT}{Wissenschaftliche Untersuchungen zum Neuen Testament}
\end{refimp}

\medskip

\begin{lstlisting}
@book{jeremias:1967,
  author = {Jeremias, Joachim},
  title = {The Prayers of Jesus},
  shorttitle = {Prayers},
  series = {Studies in Biblical Theology},
  shortseries = {SBT},
  seriesseries = {2},
  number = {6},
  location = {Naperville, IL},
  publisher = {Allenson},
  date = {1967}
}
\end{lstlisting}

\begin{biboutput}
  \samplecite{23}[123-127]{jeremias:1967}
  \samplecite{32}[126]{jeremias:1967}
  \samplebib{jeremias:1967}
  \samplebiblist{jeremias:1967}
\end{biboutput}

\begin{refimp}
  \hspace*{\bibindent}23. Joachim Jeremias, \emph{The Prayers of Jesus,}
  SBT~2/6 (Naperville, IL: Allenson, 1967), 123–27. 

  \hspace*{\bibindent}32. Jeremias, \emph{Prayers,} 126.

  \hangindent\bibindent Jeremias, Joachim. \emph{The Prayers of Jesus.}
  SBT~2/6. Naperville, IL: Allenson, 1967.

  \refbiblist{SBT}{Studies in Biblical Theology}
\end{refimp}

\subsubsection{Electronic Book}

\begin{lstlisting}
@book{reventlow:2009,
  author = {Reventlow, Henning Graf},
  title = {From the Old Testament to Origen},
  volume = {1},
  maintitle = {History of Biblical Interpretation},
  translator = {Perdue, Leo G.},
  location = {Atlanta},
  publisher = {Society of Biblical Literature},
  date = {2009},
  eprint = {Nook},
  eprinttype = {ebook}
}
\end{lstlisting}

\begin{biboutput}
  \samplemacro{\textbackslash autocite[ch.\textasciitilde
  1.3]\{reventlow:2009\}}
  \samplecite*{14}[ch.~1.3]{reventlow:2009}
  \samplemacro{\textbackslash autocite[ch.\textasciitilde
  1.3]\{reventlow:2009\}}
  \samplecite*{18}[ch.~1.3]{reventlow:2009}
  \samplebib{reventlow:2009}
\end{biboutput}

\begin{refimp}
  \hspace*{\bibindent}14. Henning Graf Reventlow, \emph{From the Old Testament
  to Origen.} Vol.~1\footnote{Should be “, vol.~1”?} of \emph{History of
  Biblical Interpretation,} trans.\@ Leo G. Perdue (Atlanta: Society of
  Biblical Literature, 2009), Nook edition, ch.~1.3.

  \hspace*{\bibindent}18. Reventlow, \emph{From the Old Testament to Origen,}
  ch.~1.3.
  
  \hangindent\bibindent Reventlow, Henning Graf. \emph{From the Old Testament
  to Origen.} Volume\footnote{Should be “Vol.” as elsewhere?} 1 of
  \emph{History of Biblical Interpretation.} Translated by Leo G. Perdue.
  Atlanta: Society of Biblical Literature, 2009. Nook edition.
\end{refimp}

\medskip

\begin{lstlisting}
@book{wright:2014,
  author = {Wright, Jacob L.},
  title = {David, King of Israel, and Caleb in Biblical Memory},
  shorttitle = {David, King of Israel},
  location = {Cambridge},
  publisher = {Cambridge University Press},
  date = {2014},
  eprint = {Kindle},
  eprinttype = {ebook}
}
\end{lstlisting}

\begin{biboutput}
  \samplemacro{\textbackslash autocite[ch.\textasciitilde 3, \textbackslash
  mkibquote\{Introducing David\}]\{wright:2014\}}
  \samplecite*{3}[ch.~3, \mkbibquote{Introducing David}]{wright:2014}
  \samplemacro{\textbackslash autocite[ch.\textasciitilde 5, \textbackslash
  mkibquote\{Evidence from Qumran\}]\{wright:2014\}}
  \samplecite*{21}[ch.~5, \mkbibquote{Evidence from Qumran}]{wright:2014}
  \samplebib{wright:2014}
\end{biboutput}

\begin{refimp}
  \hspace*{\bibindent}3. Jacob L. Wright, \emph{David, King of Israel, and
  Caleb in Biblical Memory} (Cambridge: Cambridge University Press, 2014),
  Kindle edition, ch.~3, “Introducing David.”

  \hspace*{\bibindent}21. Wright, \emph{David, King of Israel,} ch.~5,
  “Evidence from Qumran.”

  \hangindent\bibindent Jacob L. Wright, \emph{David, King of Israel, and
  Caleb in Biblical Memory.} Cambridge: Cambridge University Press, 2014.
  Kindle edition.
\end{refimp}

\begin{lstlisting}
@book{killebrew+steiner:2014,
  editor = {Killebrew, Ann E. and Steiner, Margreet},
  title = {The Oxford Handbook of the Archaeology of the Levant: c.~8000--332 BCE},
  shorttitle = {Archaeology of the Levant},
  location = {Oxford},
  publisher = {Oxford University Press},
  date = {2014},
  doi = {10.1093/oxfordhb/9780199212972.001.0001}
}
\end{lstlisting}

\begin{biboutput}
  \samplecite{53}{killebrew+steiner:2014}
  \samplecite{55}{killebrew+steiner:2014}
  \samplebib{killebrew+steiner:2014}
\end{biboutput}

\begin{refimp}
  \hspace*{\bibindent}53. Ann E. Killebrew and Margreet Steiner, eds.,
  \emph{The Oxford Handbook of the Archaeology of the Levant: c.~8000–332 BCE}
  (Oxford: Oxford University Press, 2014),
  doi:10.1093/oxfordhb/9780199212972.001.0001.

  \hspace*{\bibindent}55. Killebrew and Steiner, \emph{Archaeology of the
  Levant.}

  \hangindent\bibindent Killebrew, Ann E.\footnote{Should be “Killebrew, Ann
  E.,”?} and Margreet Steiner, eds. \emph{The Oxford Handbook of the
  Archaeology of the Levant: c.~8000–332 BCE.} Oxford: Oxford University
  Press, 2014. doi:10.1093/ oxfordhb/9780199212972.001.0001.
\end{refimp}

\begin{lstlisting}
@book{kaufman:1974,
  author = {Kaufman, Stephen},
  title = {The Akkadian Influences on Aramaic},
  series = {Assyriological Studies},
  shortseries = {AS},
  number = {19},
  location = {Chicago},
  publisher = {The Oriental Institute of the University of Chicago},
  date = {1974},
  url = {http://oi.uchicago.edu/pdf/as19.pdf}
}
\end{lstlisting}

\begin{biboutput}
  \samplecite{29}{kaufman:1974}
  \samplecite{32}[123]{kaufman:1974}
  \samplebib{kaufman:1974}
  \samplebiblist{kaufman:1974}
\end{biboutput}

\begin{refimp}
  \hspace*{\bibindent}29. Stephen Kaufman. \emph{The Akkadian Influences on
  Aramaic,} AS~19 (Chicago: The Oriental Institute of the University of
  Chicago, 1974), \nolinkurl{http://oi.uchicago.edu/pdf/as19.pdf.}

  \hspace*{\bibindent}32. Kaufman, \emph{Akkadian Influences on Aramaic,} 123.

  \hangindent\bibindent Kaufman, Stephen. \emph{The Akkadian Influences on
  Aramaic.} AS~19. Chicago: The Oriental Institute of the University of
  Chicago, 1974. \nolinkurl{http://oi.uchicago.edu/pdf/as19.pdf.}

  \refbiblist{AS}{Assyriological Studies}
\end{refimp}

\subsection{General Examples: Journal Articles, Reviews, and Dissertations}

\subsubsection{A Journal Article}

\begin{lstlisting}
@article{leyerle:1993,
  author = {Leyerle, Blake},
  title = {John Chrysostom on the Gaze},
  shorttitle = {Chrysostom},
  journaltitle = {Journal of Early Christian Studies},
  shortjournal = {JECS},
  volume = {1},
  date = {1993},
  pages = {159-174}
}
\end{lstlisting}  

\begin{biboutput}
  \samplecite{7}{leyerle:1993}
  \samplecite{23}[161]{leyerle:1993}
  \samplebib{leyerle:1993}
  \samplebiblist{leyerle:1993}
\end{biboutput}

\begin{refimp}
  \hspace*{\bibindent}7. Blake Leyerle, “John Chrysostom on the Gaze,”
  \emph{JECS} 1 (1993): 159–74.

  \hspace*{\bibindent}23. Leyerle, “John Chrysostom,” 161.

  \hangindent\bibindent Leyerle, Blake. “John Chrysostom on the Gaze.”
  \emph{JECS} 1 (1993): 159–74.

  \refbiblist{\emph{JECS}}{\emph{Journal of Early Christian Studies}}
\end{refimp}

\subsubsection{A Journal Article with Multiple Page Locations and Volumes}

\begin{lstlisting}
@article{wildberger:1965,
  author = {Wildberger, Hans},
  title = {Das Abbild Gottes: Gen 1:26--30},
  journaltitle = {Theologische Zeitschrift},
  shortjournal = {TZ},
  volume = {21},
  date = {1965},
  pages = {245-259, 481-501}
}
\end{lstlisting}

\begin{biboutput}
  \samplecite{21}{wildberger:1965}
  \samplebib{wildberger:1965}
  \samplebiblist{wildberger:1965}
\end{biboutput}

\begin{refimp}
  \hspace*{\bibindent}21. Hans Wildberger, “Das Abbild Gottes: Gen 1:26–30,”
  \emph{TZ}~21 (1965): 245–59, 481–501.

  \hangindent\bibindent Wildberger, Hans. “Das Abbild Gottes: Gen 1:26–30.”
  \emph{TZ}~21 (1965): 245–59, 481–501.

  \refbiblist{\emph{TZ}}{\emph{Theologische Zeitschrift}}
\end{refimp}

\begin{lstlisting}
@article{wellhausen:1876-1877,
  author = {Wellhausen, Julius},
  title = {Die Composition des Hexateuchs},
  journaltitle = {Jahrbuch für deutsche Theologie},
  shortjournal = {JDT},
  pages = {21 \mkbibparens{1876}: 392--450\addsemicolon\space 22 \mkbibparens{1877}: 407--479}
}
\end{lstlisting}

\begin{biboutput}
  \samplecite{24}{wellhausen:1876-1877}
  \samplebib{wellhausen:1876-1877}
  \samplebiblist{wellhausen:1876-1877}
\end{biboutput}

\begin{refimp}
  \hspace*{\bibindent}24. Julius Wellhausen, “Die Composition des Hexateuchs,”
  \emph{JDT}~21 (1876): 392–450; 22 (1877): 407–79.

  \hangindent\bibindent Wellhausen, Julius. “Die Composition des Hexateuchs.”
  \emph{JDT}~21 (1876): 392–450; 22 (1877): 407–79.

  \refbiblist{\emph{JDT}}{\emph{Jahrbuch für deutsche Theologie}}
\end{refimp}

\subsubsection{A Journal Article Republished in a Collected Volume}

\begin{lstlisting}
@article{freedman:1977,
  author = {Freedman, David Noel},
  title = {Pottery, Poetry, and Prophecy: An Essay on Biblical Poetry},
  journaltitle = {Journal of Biblical Literature},
  shortjournal = {JBL},
  volume = {96},
  date = {1977},
  pages = {5-26}
}
\end{lstlisting}

\begin{biboutput}
  \samplecite{20}[20]{freedman:1977}
  \samplebib{freedman:1977}
  \samplebiblist{freedman:1977}
\end{biboutput}

\begin{refimp}
  \hspace*{\bibindent}20. David Noel Freedman, “Pottery, Poetry, and Prophecy:
  An Essay on Biblical Poetry,” \emph{JBL}~96 (1977): 20.\footnote{Should be
  “5–26” (full page reference) as elsewhere?}

  \hangindent\bibindent Freedman, David Noel. “Pottery, Poetry, and Prophecy:
  An Essay on Biblical Poetry.” \emph{JBL}~96 (1977): 5–26.

  \refbiblist{\emph{JBL}}{\emph{Journal of Biblical Literature}}
\end{refimp}

\begin{lstlisting}
@incollection{freedman:1980,
  author = {Freedman, David Noel},
  title = {Pottery, Poetry, and Prophecy: An Essay on Biblical Poetry},
  booktitle = {Pottery, Poetry, and Prophecy: Studies in Early Hebrew Poetry},
  location = {Winona Lake, IN},
  publisher = {Eisenbrauns},
  date = {1980},
  pages = {1-22}
}
\end{lstlisting}

\begin{biboutput}
  \samplecite{20}[14]{freedman:1980}
  \samplebib{freedman:1980}
\end{biboutput}

\begin{refimp}
  \hspace*{\bibindent}20. David Noel Freedman, “Pottery, Poetry, and Prophecy:
  An Essay on Biblical Poetry,” in \emph{Pottery, Poetry, and Prophecy:
  Studies in Early Hebrew Poetry} (Winona Lake, IN: Eisenbrauns, 1980),
  14.\footnote{Should be “1–22” (full page reference) as elsewhere?}

  \hangindent\bibindent Freedman, David Noel. “Pottery, Poetry, and Prophecy:
  An Essay on Biblical Poetry.” Pages 1–22 in \emph{Pottery, Poetry, and
  Prophecy: Studies in Early Hebrew Poetry.} Winona Lake, IN: Eisenbrauns,
  1980.
\end{refimp}

\subsubsection{A Book Review}

\begin{lstlisting}
@review{teeple:1966,
  author = {Teeple, Howard M.},
  revdauthor = {Robert, André and Feuillet, André},
  revdtitle = {Introduction to the New Testament},
  journaltitle = {Journal of Bible and Religion},
  shortjournal = {JBR},
  volume = {34},
  date = {1966},
  pages = {368-370}
}
\end{lstlisting}

\begin{biboutput}
  \samplecite{8}{teeple:1966}
  \samplecite{21}[369]{teeple:1966}
  \samplebib{teeple:1966}
  \samplebiblist{teeple:1966}
\end{biboutput}

\begin{refimp}
  \hspace*{\bibindent}8. Howard M. Teeple, review of \emph{Introduction to the
  New Testament,} by André Robert and André Feuillet, \emph{JBR} 34 (1966):
  368–70.
  
  \hspace*{\bibindent}21. Teeple, review of \emph{Introduction to the New
  Testament} (by Robert and Feuillet), 369.

  \hangindent\bibindent Teeple, Howard M. Review of \emph{Introduction to the
  New Testament,} by André Robert and André Feuillet. \emph{JBR} 34 (1966):
  368–70.

  \refbiblist{\emph{JBR}}{\emph{Journal of Bible and Religion}}
\end{refimp}

\begin{lstlisting}
@review{pelikan:1992,
  author = {Pelikan, Jaroslav},
  title = {The Things That You're Liable to Read in the Bible},
  revdeditor = {Freedman, David Noel},
  revdtitle = {The Anchor Bible Dictionary},
  journaltitle = {New York Times Review of Books},
  date = {1992-12-20},
  pages = {3}
}
\end{lstlisting}

\begin{biboutput}
  \samplecite{9}{pelikan:1992}
  \samplebib{pelikan:1992}
\end{biboutput}

\begin{refimp}
  \hspace*{\bibindent}9. Jaroslav Pelikan, “The Things That You're Liable to
  Read in the Bible,” review of \emph{The Anchor Bible Dictionary,} ed.\@
  David Noel Freedman. \emph{New York Times Review of Books,} 20 December
  1992, 3.
  
  \hangindent\bibindent Pelikan, Jaroslav. “The Things That You're Liable to
  Read in the Bible,” review of \emph{The Anchor Bible Dictionary,} ed.\@
  David Noel Freedman. \emph{New York Times Review of Books,} 20 December
  1992, 3.
\end{refimp}

\begin{lstlisting}
@article{petersen:1988,
  author = {Petersen, David L.},
  title = {Hebrew Bible Textbooks: A Review Article},
  journaltitle = {Critical Review of Books in Religion},
  shortjournal = {CRBR},
  volume = {1},
  date = {1988},
  pages = {1-18}
}
\end{lstlisting}

\begin{biboutput}
  \samplecite{7}{petersen:1988}
  \samplecite{14}[8]{petersen:1988}
  \samplebib{petersen:1988}
  \samplebiblist{petersen:1988}
\end{biboutput}

\begin{refimp}
  \hspace*{\bibindent}7. David Petersen, “Hebrew Bible Textbooks: A Review
  Article,” \emph{CRBR} 1 (1988): 1–18.

  \hspace*{\bibindent}14. Petersen, “Hebrew Bible Textbooks,” 8.

  \hangindent\bibindent Petersen, David. “Hebrew Bible Textbooks: A Review
  Article.” \emph{CRBR} 1 (1988): 1–18.

  \refbiblist{\emph{CRBR}}{\emph{Critical Review of Books in Religion}}
\end{refimp}

\subsubsection{An Unpublished Dissertation or Thesis}

\begin{lstlisting}
@thesis{klosinski:1988,
  author = {Klosinski, Lee E.},
  title = {Meals in Mark},
  type = {phdthesis},
  institution = {The Claremont Graduate School},
  date = {1988}
}
\end{lstlisting}  

\begin{biboutput}
  \samplecite{21}[22-44]{klosinski:1988}
  \samplecite{26}[23]{klosinski:1988}
  \samplebib{klosinski:1988}
\end{biboutput}

\begin{refimp}
  \hspace*{\bibindent}21. Lee E. Klosinski, “Meals in Mark” (PhD diss., The
  Claremont Graduate School, 1988), 22–44.

  \hspace*{\bibindent}26. Klosinski, “Meals in Mark,” 23.

  \hangindent\bibindent Klosinski, Lee. E. “Meals in Mark.” PhD diss., The
  Claremont Graduate School, 1988.
\end{refimp}

\subsubsection{An Article in an Encyclopaedia or a Dictionary}

\paragraph{An Article in a Multivolume Encyclopaedia or Dictionary}

\begin{lstlisting}
@mvreference{IDB,
  editor = {Buttrick, George A.},
  title = {The Interpreter's Dictionary of the Bible},
  shorttitle = {IDB},
  volumes = {4},
  location = {New York},
  publisher = {Abingdon},
  date = {1962},
  shorthand = {IDB}
}

@inreference{stendahl:1962,
  author = {Stendahl, Krister},
  title = {Biblical Theology, Contemporary},
  shorttitle = {Biblical Theology},
  volume = {1},
  pages = {418-432},
  crossref = {IDB},
  xref = {IDB}
}
\end{lstlisting}  

\begin{biboutput}
  \samplecite{33}{stendahl:1962}
  \samplecite{36}[419]{stendahl:1962}
  \samplebib{stendahl:1962}
  \samplemacro{\textbackslash usepackage[style=sbl,fullbibrefs]\{biblatex\} \\
  \textbackslash printbibliography}
  \toggletrue{fullbibrefs}
  \samplebib*{stendahl:1962}
  \togglefalse{fullbibrefs}
  \samplebiblist{IDB}
\end{biboutput}

\begin{refimp}
  \hspace*{\bibindent}33. Krister Stendahl, “Biblical Theology, Contemporary,”
  \emph{IDB} 1:418–32.
  
  \hspace*{\bibindent}36. Stendahl, “Biblical Theology,” 1:419.
  
  Stendahl, Krister. “Biblical Theology, Contemporary.” \emph{IDB} 1:418–32.

  \refbiblist{\emph{IDB}}{\emph{The Interpreter's Dictionary of the Bible.}
  Edited by George A. Buttrick. 4~vols. New York: Abingdon, 1962}
\end{refimp}

\paragraph{An Article in a Single-Volume Encyclopaedia or Dictionary}

\begin{lstlisting}
@reference{DOTP,
  editor = {Alexander, T. Desmond and Baker, David W.},
  title = {Dictionary of the Old Testament: Pentateuch},
  shorttitle = {DOTP},
  location = {Downers Grove, IL},
  publisher = {InterVarsity},
  date = {2003},
  shorthand = {DOTP}
}

@inreference{olson:2003,
  author = {Olson, Dennis T.},
  title = {Numbers, Book of},
  shorttitle = {Numbers},
  pages = {611-618},
  crossref = {DOTP},
  xref = {DOTP}
}
\end{lstlisting}  

\begin{biboutput}
  \samplecite{33}{olson:2003}
  \samplecite{36}[612]{olson:2003}
  \samplebib{olson:2003}
  \samplebiblist{DOTP}
\end{biboutput}

\subsubsection{An Article in a Lexicon or Theological Dictionary}

\begin{lstlisting}
@mvlexicon{NIDNTT,
  editor = {Brown, Colin},
  title = {New International Dictionary of New Testament Theology},
  shorttitle = {NIDNTT},
  volumes = {4},
  location = {Grand Rapids},
  publisher = {Zondervan},
  date = {1975/1985},
  shorthand = {NIDNTT}
}

@inlexicon{dahn+liefeld:see+vision+eye,
  author = {Dahn, Karl and Liefeld, Walter L.},
  title = {See, Vision, Eye},
  volume = {3},
  pages = {511-521},
  xref = {NIDNTT}
}
\end{lstlisting}

\begin{biboutput}
  \samplecite{3}{dahn+liefeld:see+vision+eye}
  \samplebib{NIDNTT}
  \samplebiblist{NIDNTT}
\end{biboutput}

\begin{refimp}
  \hspace*{\bibindent}3. Karl Dahn and Walter L. Liefeld, “See, Vision, Eye,”
  \emph{NIDNTT} 3:511–21.

  \hangindent\bibindent Brown, Colin, ed. \emph{New International Dictionary
  of New Testament Theology.} 4 vols. Grand Rapids: Zondervan, 1975–1985.

  \refbiblist{\emph{NIDNTT}}{\emph{New International Dictionary of New
  Testament Theology.} Edited by Colin Brown. 4 vols. Grand Rapids: Zondervan,
  1975–1985}
\end{refimp}

\medskip

\begin{lstlisting}
@mvlexicon{TDNT,
  editor = {Kittel, Gerhard and Friedrich, Gerhard},
  title = {Theological Dictionary of the New Testament},
  shorttitle = {TDNT},
  translator = {Bromiley, Geoffrey W.},
  volumes = {10},
  location = {Grand Rapids},
  publisher = {Eerdmans},
  date = {1964/1976},
  shorthand = {TDNT}
}

@inlexicon{beyer:diakoneo+diakonia+ktl,
  author = {Beyer, Hermann W.},
  title = {\textgreek{διακονέω, διακονία, κτλ}},
  volume = {2},
  pages = {81-93},
  xref = {TDNT}
}
\end{lstlisting}
  
\begin{biboutput}
  \samplecite{6}{beyer:diakoneo+diakonia+ktl}
  \samplecite{25}[83]{beyer:diakoneo+diakonia+ktl}
  \samplebib{TDNT}
  \samplebiblist{TDNT}
\end{biboutput}

\begin{refimp}
  \hspace*{\bibindent}6. Hermann W. Beyer, “\textgreek{διακονέω, διακονία, κτλ},”
  \emph{TDNT} 2:81–93.

  \hspace*{\bibindent}25. Beyer, \emph{TDNT} 2:83.

  \hangindent\bibindent Kittel, Gerhard, and Gerhard Friedrich, eds.
  \emph{Theological Dictionary of the New Testament.} Translated by Geoffrey
  W. Bromiley. 10 vols. Grand Rapids: Eerdmans, 1964–1976.

  \refbiblist{\emph{TDNT}}{\emph{Theological Dictionary of the New Testament.}
    Edited by Gerhard Kittel and Gerhard Friedrich. Translated by Geoffrey W.
  Bromiley. 10 vols. Grand Rapids: Eerdmans, 1964–1976}
\end{refimp}

\medskip

\begin{lstlisting}
@mvlexicon{TLNT,
  author = {Spicq, Ceslas},
  title = {Theological Lexicon of the New Testament},
  shorttitle = {TLNT},
  editor = {Ernest, James D.},
  translator = {Ernest, James D.},
  volumes = {3},
  location = {Peabody, MA},
  publisher = {Hendrickson},
  date = {1994},
  shorthand = {TLNT}
}

@inlexicon{spicq:atakteo+ataktos+ataktos,
  author = {Spicq, Ceslas},
  title = {\textgreek{ἀτακτέω, ἄτακτος, ἀτάκτως}},
  volume = {1},
  pages = {223-224},
  xref = {TLNT}
}
\end{lstlisting}

\begin{biboutput}
  \samplecite{7}{spicq:atakteo+ataktos+ataktos}
  \samplebib{TLNT}
  \samplebiblist{TLNT}
\end{biboutput}

\begin{refimp}
  \hspace*{\bibindent}7. Ceslas Spicq, “\textgreek{ἀτακτέω, ἄτακτος, ἀτάκτως},”
  \emph{TLNT} 1:223–24.

  \hangindent\bibindent Spicq, Ceslas. \emph{Theological Lexicon of the New
  Testament.} Translated and edited by James D. Ernest. 3 vols. Peabody, MA:
  Hendrickson, 1994.

  \refbiblist{\emph{TLNT}}{\emph{Theological Lexicon of the New Testament.}
    Ceslas Spicq. Translated and edited by James D. Ernest. 3 vols. Peabody,
  MA: Hendrickson, 1994}
\end{refimp}

\medskip

\begin{lstlisting}
@mvlexicon{TLNT,
  author = {Spicq, Ceslas},
  title = {Theological Lexicon of the New Testament},
  shorttitle = {TLNT},
  editor = {Ernest, James D.},
  translator = {Ernest, James D.},
  volumes = {3},
  location = {Peabody, MA},
  publisher = {Hendrickson},
  date = {1994},
  shorthand = {TLNT}
}

@inlexicon{spicq:amoibe,
  author = {Spicq, Ceslas},
  title = {\textgreek{ἀμοιβή}},
  volume = {1},
  pages = {95-96},
  xref = {TLNT}
}
\end{lstlisting}

\begin{biboutput}
  \samplecite{143}{spicq:amoibe}
  \samplecite{147}[95]{spicq:amoibe}
  \samplebib{TLNT}
  \samplebiblist{TLNT}
\end{biboutput}

\begin{refimp}
  143. Ceslas Spicq, “\textgreek{ἀμοιβή},” \emph{TLNT} 1:95–96.

  147. Spicq, \emph{TLNT} 1:95.

  \hangindent\bibindent Spicq, Ceslas. \emph{Theological Lexicon of the New
  Testament.} Translated and edited by James D. Ernest. 3 vols. Peabody, MA:
  Hendrickson, 1994.

  \refbiblist{\emph{TLNT}}{\emph{Theological Lexicon of the New Testament.}
    Ceslas Spicq. Translated and edited by James D. Ernest. 3 vols. Peabody,
  MA: Hendrickson, 1994}
\end{refimp}

\medskip

\begin{lstlisting}
@mvlexicon{TDNT,
  editor = {Kittel, Gerhard and Friedrich, Gerhard},
  title = {Theological Dictionary of the New Testament},
  shorttitle = {TDNT},
  translator = {Bromiley, Geoffrey W.},
  volumes = {10},
  location = {Grand Rapids},
  publisher = {Eerdmans},
  date = {1964/1976},
  shorthand = {TDNT}
}

@inlexicon{beyer:diakoneo,
  author = {Beyer, Hermann W.},
  title = {\textgreek{διακονέω}},
  volume = {2},
  pages = {81-87},
  xref = {TDNT}
}
\end{lstlisting}

\begin{biboutput}
  \samplecite{23}{beyer:diakoneo}
  \samplebib{TDNT}
  \samplebiblist{TDNT}
\end{biboutput}

\begin{refimp}
  \hspace*{\bibindent}23. Hermann W. Beyer, “\textgreek{διακονέω},” \emph{TDNT}
  2:81–87.

  \hangindent\bibindent Kittel, Gerhard, and Gerhard Friedrich, eds.
  \emph{Theological Dictionary of the New Testament.} Translated by Geoffrey
  W. Bromiley. 10 vols. Grand Rapids: Eerdmans, 1964–1976.

  \refbiblist{\emph{TDNT}}{\emph{Theological Dictionary of the New Testament.}
  Edited by Gerhard Kittel and Gerhard Friedrich. Translated by Geoffrey W.
  Bromiley. 10 vols. Grand Rapids: Eerdmans, 1964–1976}
\end{refimp}

\medskip

\begin{lstlisting}
@mvlexicon{NIDNTT,
  editor = {Brown, Colin},
  title = {New International Dictionary of New Testament Theology},
  shorttitle = {NIDNTT},
  volumes = {4},
  location = {Grand Rapids},
  publisher = {Zondervan},
  date = {1975/1985},
  shorthand = {NIDNTT}
}

@inlexicon{dahn:horao,
  author = {Dahn, Karl},
  title = {\textgreek{ὁράω}},
  volume = {3},
  pages = {511-518},
  xref = {NIDNTT}
}
\end{lstlisting}

\begin{biboutput}
  \samplecite{26}{dahn:horao}
  \samplecite{29}[511]{dahn:horao}
  \samplebib{NIDNTT}
  \samplebiblist{NIDNTT}
\end{biboutput}

\begin{refimp}
  \hspace*{\bibindent}26. Karl Dahn, “\textgreek{ὁράω},” \emph{NIDNTT} 3:511–18.

  \hspace*{\bibindent}29. Dahn, \emph{NIDNTT} 3:511.

  \hangindent\bibindent Brown, Colin, ed. \emph{New International Dictionary
  of New Testament Theology.} 4 vols. Grand Rapids: Zondervan, 1975–1985.

  \refbiblist{\emph{NIDNTT}}{\emph{New International Dictionary of New
  Testament Theology.} Edited by Colin Brown. 4 vols. Grand Rapids:
  Zondervan, 1975–1978}
\end{refimp}

\subsubsection{A Paper Presented at a Professional Society}

\begin{lstlisting}
@misc{SBL,
  note = {Society of Biblical Literature},
  shorthand = {SBL},
  options = {skipbib}
}

@conferencepaper{niditch:1994,
  author = {Niditch, Susan},
  title = {Oral Culture and Written Documents},
  shorttitle = {Oral Culture},
  eventtitle = {the Annual Meeting of the New England Region of the \citeshorthand{SBL}},
  venue = {Worcester, MA},
  eventdate = {1994-03-25}
}
\end{lstlisting}

\begin{biboutput}
  \samplecite{31}[13-17]{niditch:1994}
  \samplecite{35}[14]{niditch:1994}
  \samplebib{niditch:1994}
  \samplebiblist{SBL}
\end{biboutput}

\begin{refimp}
  \hspace*{\bibindent}31. Susan Niditch, “Oral Culture and Written Documents”
  (paper presented at the Annual Meeting of the New England Region of the SBL,
  Worcester, MA, 25 March 1994), 13–17.

  \hspace*{\bibindent}35. Niditch, “Oral Culture,” 14.

  \hangindent\bibindent Niditch, Susan. “Oral Culture and Written Documents.”
  Paper presented at the Annual Meeting of the New England Region of the SBL.
  Worcester, MA, 25 March 1994.

  \refbiblist{SBL}{Society of Biblical Literature}
\end{refimp}

\subsubsection{An Article in a Magazine}

\begin{lstlisting}
@article{saldarini:1998,
  author = {Saldarini, Anthony J.},
  title = {Babatha's Story},
  journaltitle = {Biblical Archaeology Review},
  shortjournal = {BAR},
  volume = {24},
  number = {2},
  date = {1998},
  pages = {23-33, 36-37, 72-74}
}
\end{lstlisting}  

\begin{biboutput}
  \samplecite{8}{saldarini:1998}
  \samplecite{27}[28]{saldarini:1998}
  \samplebib{saldarini:1998}
  \samplebiblist{saldarini:1998}
\end{biboutput}

\begin{refimp}
  \hspace*{\bibindent}8.Anthony J. Saldarini, “Babatha’s Story,”
  \emph{BAR}~24.2 (1998): 28–33, 36–37, 72–74.

  \hspace*{\bibindent}27. Saldarini, “Babatha’s Story,” 28.

 \hangindent\bibindent Saldarini, Anthony J. “Babatha’s Story.”
 \emph{BAR}~24.2 (1998): 28–33, 36–37, 72–74.

 \refbiblist{\emph{BAR}}{\emph{Biblical Archaeology Review}}
\end{refimp}

\subsubsection{An Electronic Journal Article}

\begin{lstlisting}
@article{springer:2014,
  author = {Springer, Carl P. E.},
  title = {Of Roosters and \mkbibemph{Repetitio}: Ambrose's \mkbibemph{Aeterne rerum conditor}},
  journaltitle = {Vigiliae Christianae},
  shortjournal = {VC},
  volume = {68},
  date = {2014},
  pages = {155-177},
  doi = {10.1163/15700720-12341158}
}
\end{lstlisting}

\begin{biboutput}
  \samplecite{43}{springer:2014}
  \samplecite{45}[158]{springer:2014}
  \samplebib{springer:2014}
  \samplebiblist{springer:2014}
\end{biboutput}

\begin{refimp}
  \hspace*{\bibindent}43. Carl P. E. Springer, “Of Roosters and
  \emph{Repetitio}: Ambrose’s \emph{Aeterne rerum conditor},” \emph{VC}~68
  (2014): 155–77, \nolinkurl{doi:10.1163/15700720-12341158}.

  \hspace*{\bibindent}45. Springer, “Of Roosters and \emph{Repetitio},” 158.

  \hangindent\bibindent Springer, Carl P. E. “Of Roosters and
  \emph{Repetitio}: Ambrose’s \emph{Aeterne rerum conditor}.” \emph{VC}~68
  (2014): 155–77. \nolinkurl{doi:10.1163/15700720-12341158}.

  \refbiblist{\emph{VC}}{\emph{Vigiliae Christianae}}
\end{refimp}

\begin{lstlisting}
@article{truehart:1996,
  author = {Truehart, Charles},
  title = {Welcome to the Next Church},
  shorttitle = {Next Church},
  url = {http://www.theatlantic.com/atlantic/issues/96aug/nxtchrch/nxtchrch.htm},
  journaltitle = {Atlantic Monthly},
  volume = {278},
  date = {1996-08},
  pages = {37-58}
}
\end{lstlisting}

\begin{biboutput}
  \samplecite{8}{truehart:1996}
  \samplecite{12}[37]{truehart:1996}
  \samplebib{truehart:1996}
\end{biboutput}

\begin{refimp}
  \hspace*{\bibindent}8. Charles Truehart, “Welcome to the Next Church,”
  \emph{Atlantic Monthly} 278 (August 1996): 37–58,
  \nolinkurl{http://www.theatlantic.com/past/docs/issues/96aug/nxtchrch/nxtchrch.htm}.

  \hspace*{\bibindent}12. Truehart, “Next Church,” 37.
 
  \sloppy\hangindent\bibindent Truehart, Charles. “Welcome to the Next Church.”
  \emph{Atlantic Monthly} 278 (August 1996): 37–58.
  \nolinkurl{http://www.theatlantic.com/past/docs/issues/96aug/nxtchrch/nxtchrch.htm}.
\end{refimp}

\begin{lstlisting}
@article{kirk:2007,
  author = {Kirk, Alan},
  title = {Karl Polanyi, Marshall Sahlins, and the Study of Ancient Social Relations},
  shorttitle = {Karl Polanyi},
  journaltitle = {Journal of Biblical Literature},
  shortjournal = {JBL},
  volume = {126},
  date = {2007},
  pages = {182-191},
  doi = {10.2307/27638428},
  url = {http://www.jstor.org/stable/27638428}
}
\end{lstlisting}

\begin{biboutput}
  \samplecite{31}{kirk:2007}
  \samplecite{35}[186]{kirk:2007}
  \samplebib{kirk:2007}
  \samplebiblist{kirk:2007}
\end{biboutput}

\begin{refimp}
  \hspace*{\bibindent}31. Alan Kirk, “Karl Polanyi, Marshall Sahlins, and the
  Study of Ancient Social Relations,” \emph{JBL}~126 (2007): 182–91,
  \nolinkurl{doi:10.2307/27638428},
  \nolinkurl{http://www.jstor.org/stable/27638428}.

  \hspace*{\bibindent}35. Kirk, “Karl Polanyi,” 186.

  \hangindent\bibindent Alan Kirk.\footnote{Should be “Kirk, Alan.”?} “Karl
  Polanyi, Marshall Sahlins, and the Study of Ancient Social Relations,”
  \emph{JBL}~126 (2007): 182–91. \nolinkurl{doi:10.2307/27638428}.
  \nolinkurl{http://www.jstor.org/stable/27638428}.

  \refbiblist{\emph{JBL}}{\emph{Journal of Biblical Literature}}
\end{refimp}

\subsection{Special Examples}

\subsubsection{Texts from the Ancient Near East}

\paragraph{Citing \textsl{COS}}

\begin{lstlisting}
@mvcollection{COS,
  editor = {Hallo, William W.},
  title = {The Context of Scripture},
  shorttitle = {COS},
  volumes = {3},
  location = {Leiden},
  publisher = {Brill},
  date = {1997/2002},
  shorthand = {COS},
  options = {skipbib}
}

@collection{hallo:1997,
  editor = {Hallo, William W.},
  title = {Canonical Compositions from the Biblical World},
  shorttitle = {COS},
  maintitle = {The Context of Scripture},
  volume = {1},
  location = {Leiden},
  publisher = {Brill},
  date = {1997},
  shorthand = {COS},
  xref = {COS},
  options = {skipbiblist}
}

@ancienttext{greathymnaten,
  entrysubtype = {COS},
  title = {The Great Hymn to the Aten},
  shorttitle = {Great Hymn to the Aten},
  translator = {Lichtheim, Miriam},
  volume = {1},
  part = {26},
  pages = {44-46},
  related = {hallo:1997},
  relatedoptions = {usevolume=false,skipbiblist}
}
\end{lstlisting}

\begin{biboutput}
  \samplecite{7}{greathymnaten}
  \samplecite{11}{greathymnaten}
  \samplebib{hallo:1997}
  \samplebiblist{COS}
\end{biboutput}

\begin{refimp}
  \hspace*{\bibindent}7. “The Great Hymn to the Aten,” trans.\@ Miriam Lichtheim
  (\emph{COS} 1.26:44–46).

  \hspace*{\bibindent}11. “Great Hymn to the Aten,” \emph{COS} 1.26:44–46.

  \hangindent\bibindent Hallo, William W., ed. \emph{Canonical Compositions
  from the Biblical World.} Vol.~1 of \emph{The Context of Scripture.} Leiden:
  Brill, 1997.

  \refbiblist{\emph{COS}}{\emph{The Context of Scripture.} Edited by William
  W. Hallo. 3 vols. Leiden: Brill, 1997–2002}
\end{refimp}

\paragraph{Citing Other Texts}

\begin{lstlisting}
@collection{ANET,
  editor = {Pritchard, James B.},
  title = {Ancient Near Eastern Texts Relating to the Old Testament},
  shorttitle = {ANET},
  edition = {3},
  location = {Princeton},
  publisher = {Princeton University Press},
  date = {1969},
  shorthand = {ANET}
}

@ancienttext{suppiluliumas,
  title = {Suppiluliumas and the Egyptian Queen},
  translator = {Goetz, Albrecht},
  related = {ANET},
}
\end{lstlisting}

\begin{biboutput}
  \samplecite{16}[319]{suppiluliumas}
  \samplebib{ANET}
  \samplebiblist{ANET}
\end{biboutput}

\begin{refimp}
  \hspace*{\bibindent}16. “Suppiluliumas and the Egyptian Queen,” trans.\@
  Albrecht Goetze (\emph{ANET,} 319).

  \hangindent\bibindent Pritchard, James B., ed. \emph{Ancient Near Eastern
  Texts Relating to the Old Testament.} 3rd~ed. Princeton: Princeton
  University Press, 1969.

  \refbiblist{\emph{ANET}}{\emph{Ancient Near Eastern Texts Relating to the
  Old Testament.} Edited by James B. Pritchard. 3rd~ed. Princeton: Princeton
  University Press, 1969}
\end{refimp}

\medskip

\begin{lstlisting}
@book{dalley:1991,
  author = {Dalley, Stephanie},
  title = {Myths from Mesopotamia},
  location = {Oxford},
  publisher = {Oxford University Press},
  date = {1991}
}

@ancienttext{erraandishum,
  title = {Erra and Ishum},
  pages = {282-315},
  related = {dalley:1991},
}
\end{lstlisting}

\begin{biboutput}
  \samplecite{5}{erraandishum}
  \samplebib{dalley:1991}
\end{biboutput}

\begin{refimp}
  \hspace*{\bibindent}5. “Erra and Ishum” (Stephanie Dalley, \emph{Myths from
  Mesopotamia} [Oxford: Oxford University Press, 1991], 282–315).
  
  \hangindent\bibindent Dalley, Stephanie. \emph{Myths from Mesopotamia.}
  Oxford: Oxford University Press, 1991.
\end{refimp}

\medskip

\begin{lstlisting}
@book{foster:1993,
  author = {Foster, Benjamin},
  title = {Before the Muses: An Anthology of Akkadian Literature},
  shorttitle = {Muses},
  volume = {1},
  location = {Bethesda, MD},
  publisher = {CDL},
  date = {1993}
}

@ancienttext{erraandishum:foster,
  title = {Erra and Ishum},
  pages = {771-805},
  related = {foster:1993},
  relatedoptions = {usevolume=false}
}
\end{lstlisting}

\begin{biboutput}
  \samplecite{5}{erraandishum:foster}
  \samplebib{foster:1993}
\end{biboutput}

\begin{refimp}
  \hspace*{\bibindent}5. “Erra and Ishum” (Benjamin Foster, \emph{Before the
  Muses: An Anthology of Akkadian Literature} [Bethesda, MD: CDL, 1993],
  1:771–805).
  
  \hangindent\bibindent Foster, Benjamin. \emph{Before the Muses: An Anthology of
  Akkadian Literature.} Vol.~1. Bethesda, MD: CDL, 1993.
\end{refimp}

\medskip

\begin{lstlisting}
@mvbook{AEL,
  author = {Lichtheim, Miriam},
  title = {Ancient Egyptian Literature},
  shorttitle = {AEL},
  volumes = {3},
  location = {Berkeley},
  publisher = {University of California Press},
  date = {1971/1980},
  shorthand = {AEL},
  options = {skipbib}
}

@book{lichtheim:1976,
  author = {Lichtheim, Miriam},
  title = {Ancient Egyptian Literature},
  shorttitle = {AEL},
  volume = {2},
  location = {Berkeley},
  publisher = {University of California Press},
  date = {1976},
  shorthand = {AEL},
  ref = {AEL},
  options = {skipbiblist}
}

@ancienttext{doomedprince,
  title = {The Doomed Prince},
  pages = {200-203},
  related = {lichtheim:1976},
  relatedoptions = {shorthand=short,usevolume=false,skipbiblist}
}
\end{lstlisting}

\begin{biboutput}
  \samplecite{34}{doomedprince}
  \samplecite{36}[200-203]{doomedprince}
  \samplebib{lichtheim:1976}
  \samplebiblist{AEL}
\end{biboutput}

\begin{refimp}
  \hspace*{\bibindent}34. “The Doomed Prince” (Miriam Lichtheim, \emph{Ancient
  Egyptian Literature} [Berkeley: University of California Press, 1976],
  2:200–203).\footnote{Should be “200–3”?}
  
  \hspace*{\bibindent}36. “The Doomed Prince” (\emph{AEL}
  2:200–203).\footnote{Should be “200–3”?}
  
  \hangindent \bibindent Lichtheim, Miriam. \emph{Ancient Egyptian
  Literature.} Vol~2. Berkeley: University of California Press, 1976.

  \refbiblist{\emph{AEL}}{\emph{Ancient Egyptian Literature.} Miriam
  Lichtheim. 3 vols. Berkeley: University of California Press, 1971–1980}
\end{refimp}

\medskip

\begin{lstlisting}
@book{hoffner:1990,
  author = {Hoffner, Jr., Harry A.},
  title = {Hittite Myths},
  editor = {Beckman, Gary M.},
  series = {Writings from the Ancient World},
  shortseries = {WAW},
  number = {2},
  location = {Atlanta},
  publisher = {Scholars Press},
  date = {1990}
}

@ancienttext{disappearanceofsungod,
  title = {The Disappearance of the Sun God},
  related = {hoffner:1990}
}
\end{lstlisting}

\begin{biboutput}
  \samplemacro{\textbackslash autocite[(§3 \textbackslash{}mkbibparens\{A I
  11--17\})26]\{disappearanceofsungod\}}
  \samplecite*{12}[(§3 \mkbibparens{A I 11--17})26]{disappearanceofsungod}
  \samplebib{hoffner:1990}
  \samplebiblist{hoffner:1990}
\end{biboutput}

\begin{refimp}
  \hspace*{\bibindent}12. “The Disappearance of the Sun God,” §3 (A I 11–17)
  (Harry A. Hoffner Jr., \emph{Hittite Myths} [ed.\@ Gary M. Beckman;
  WAW~2;\footnote{Editor, series, and number should be outside of
  parentheses?} Atlanta: Scholars Press, 1990], 26).
  
  \hangindent\bibindent Hoffner, Harry A., Jr. \emph{Hittite Myths.} Edited by
  Gary M. Beckman. WAW~2. Atlanta: Scholars Press, 1990.

  \refbiblist{WAW}{Writings from the Ancient World}
\end{refimp}

\medskip

\begin{lstlisting}
@mvbook{RIMA,
  entrysubtype = {RIMA},
  author = {Grayson, Albert Kirk},
  title = {Assyrian Rulers of the Early First Millennium BC \mkbibparens{1114--859 BC}},
  series = {The Royal Inscriptions of Mesopotamia, Assyrian Periods},
  shortseries = {RIMA},
  number = {2},
  location = {Toronto},
  publisher = {University of Toronto Press},
  date = {1991},
  shorthand = {RIMA},
  options = {skipbiblistshorthand}
}

@ancienttext{ashurinscription,
  entrysubtype = {inscription},
  title = {Ashur Inscription},
  pages = {143-144},
  related = {RIMA},
  relatedoptions = {skipbiblistshorthand,shorthand=short}
}
\end{lstlisting}

\begin{biboutput}
  \samplemacro{(\textbackslash autocite[obv.\textbackslash{} lines
  10--17)]\{ashurinscription\}}
  \samplecite*{32}[(obv.\ lines 10--17)]{ashurinscription}
  \samplemacro{(\textbackslash autocite[obv.\textbackslash{} lines
  10--17)]\{ashurinscription\}}
  \samplecite*{34}[(obv.\ lines 10--17)]{ashurinscription}
  \samplebib{RIMA}
  \samplebiblist{series-RIMA}
\end{biboutput}

\begin{refimp}
  \hspace*{\bibindent}32. Ashur Inscription, obv.\ lines 10–17 (Albert Kirk
  Grayson, \emph{Assyrian Rulers of the Early First Millennium BC [1114–859
  BC],} RIMA~2 [Toronto: University of Toronto Press, 1991], 143–44).

  \hspace*{\bibindent}34. Ashur Inscription, obv.\ lines 10–17 (RIMA
  2:143–44).

  \hangindent\bibindent Grayson, Albert Kirk. \emph{Assyrian Rulers of the
  Early First Millennium BC (1114–859 BC).} RIMA~2. Toronto: University of
  Toronto Press, 1991.

  \refbiblist{RIMA}{The Royal Inscriptions of Mesopotamia, Assyrian Periods}
\end{refimp}

\medskip

\begin{lstlisting}
@book{ABC,
  author = {Grayson, Albert Kirk},
  title = {Assyrian and Babylonian Chronicles},
  shorttitle = {ABC},
  series = {Texts from Cuneiform Sources},
  shortseries = {TCS},
  number = {5},
  location = {Locust Valley, NY},
  publisher = {Augustin},
  date = {1975},
  shorthand = {ABC}
}

@ancienttext{esarhaddonchronicle,
  entrysubtype = {chronicle},
  title = {Esarhaddon Chronicle},
  related = {ABC},
  relatedoptions = {shorthand=short}
}
\end{lstlisting}

\begin{biboutput}
  \samplecite{33}[(lines 3--4)125]{esarhaddonchronicle}
  \samplecite{34}[(lines 3--4)125]{esarhaddonchronicle}
  \samplebib{ABC}
  \samplebiblist{ABC}
  \samplebiblist*{series-ABC}
\end{biboutput}

\begin{refimp}
  \hspace*{\bibindent}33. Esarhaddon Chronicle, lines 3–4 (Albert Kirk
  Grayson, \emph{Assyrian and Babylonian Chronicles,} TCS [Locust Valley, NY:
  Augustin, 1975], 125).

  \hspace*{\bibindent}33. Esarhaddon Chronicle, lines 3–4 (\emph{ABC,} 125).

  \hangindent\bibindent Grayson, Albert Kirk. \emph{Assyrian and Babylonian
  Chronicles.} TCS. Locust Valley, NY: Augustin, 1975.

  \refbiblist{\emph{ABC}}{\emph{Assyrian and Babylonian Chronicles.} Albert K.
  Grayson. TCS~5. Locust Valley, NY: Augustin, 1975}

  \refbiblist{TCS}{Texts from Cuneiform Sources}
\end{refimp}

\medskip

\begin{lstlisting}
@book{ARM1,
  author = {Dossin, Georges},
  title = {Lettres},
  series = {Archives royales de Mari},
  shortseries = {ARM},
  number = {1},
  origdate = {1946},
  location = {Paris},
  publisher = {Geuthner},
  date = {1967},
  shorthand = {ARM},
  options = {skipbiblistshorthand}
}
\end{lstlisting}

\begin{biboutput}
  \samplecite{45}[1.3]{ARM1}
  \samplebib{ARM1}
  \samplebiblist{series-ARM1}
\end{biboutput}

\begin{refimp}
  \hspace*{\bibindent}45. ARM 1.3.

  \hangindent\bibindent Dossin, Georges. \emph{Lettres.} ARM~1. 1946. Repr.,
  Paris: Geuthner, 1967.

  \refbiblist{ARM}{Archives royales de Mari}
\end{refimp}

\medskip

\begin{lstlisting}
@book{ARMT1,
  author = {Dossin, Georges},
  title = {Correspondance de Šamši-Addu et de ses fils},
  series = {Archives royales de Mari, transcrite et traduite},
  shortseries = {ARMT},
  number = {1},
  location = {Paris},
  publisher = {Imprimerei nationale},
  date = {1950},
  shorthand = {ARMT},
  options = {skipbiblistshorthand}
}
\end{lstlisting}

\begin{biboutput}
  \samplecite{45}[1.3]{ARMT1}
  \samplebib{ARMT1}
  \samplebiblist{series-ARMT1}
\end{biboutput}

\begin{refimp}
  \hspace*{\bibindent}45. ARMT 1.3.

  \hangindent\bibindent Georges Dossin,\footnote{Should be “Dossin,
  Georges.”?} \emph{Correspondance de Šamši-Addu et de ses fils.} ARMT~1.
  Paris: Imprimerei nationale, 1950.

  \refbiblist{ARMT}{Archives royales de Mari, transcrite et traduite}
\end{refimp}

\subsubsection{Loeb Classical Library (Greek and Latin)}

\begin{lstlisting}
@mvbook{josephus,
  title = {Josephus},
  translator = {Thackeray, Henry St.\@ J. and others},
  volumes = {10},
  series = {Loeb Classical Library},
  shortseries = {LCL},
  location = {Cambridge},
  publisher = {Harvard University Press},
  date = {1926/1965}
}

@classictext{josephus:ant,
  author = {Josephus},
  title = {Ant\adddot},
  xref = {josephus}
}
\end{lstlisting}

\begin{biboutput}
  \sampleparencite[2.233-235]{josephus:ant}
  \samplecite{1}[2.233-235]{josephus:ant}
  \samplebib{josephus}
  \samplebiblist{josephus}
\end{biboutput}

\begin{refimp}
  (Josephus, \emph{Ant.} 2.233–235)

  \hspace*{\bibindent}1. Josephus, \emph{Ant.}\ 2.233–235.

  \hangindent\bibindent \emph{Josephus.} Translated by Henry St.\@ J.
  Thackeray et al. 10 vols. LCL. Cambridge: Harvard University Press,
  1926–1965.

  \refbiblist{LCL}{Loeb Classical Library}
\end{refimp}

\medskip

\begin{lstlisting}
@mvbook{tacitus,
  author = {Tacitus},
  title = {The Histories and The Annals},
  translator = {Clifford H. Moore and John Jackson},
  volumes = {4},
  series = {Loeb Classical Library},
  shortseries = {LCL},
  location = {Cambridge},
  publisher = {Harvard University Press},
  date = {1937}
}

@classictext{tacitus:ann,
  author = {Tacitus},
  title = {Ann\adddot},
  xref = {tacitus}
}
\end{lstlisting}

\begin{biboutput}
  \samplecite{4}[15.18-19]{tacitus:ann}
  \samplebib{tacitus}
  \samplebiblist{tacitus}
\end{biboutput}

\begin{refimp}
  \hspace*{\bibindent}4. Tacitus, \emph{Ann.}\ 15.18–19.

  \hangindent\bibindent Tacitus. \emph{The Histories and The Annals.}
  Translated by Clifford H. Moore and John Jackson. 4 vols. LCL. Cambridge:
  Harvard University Press, 1937.

  \refbiblist{LCL}{Loeb Classical Library}
\end{refimp}

\medskip

\begin{lstlisting}
@mvbook{josephus,
  title = {Josephus},
  translator = {Thackeray, Henry St.\@ J. and others},
  volumes = {10},
  series = {Loeb Classical Library},
  shortseries = {LCL},
  location = {Cambridge},
  publisher = {Harvard University Press},
  date = {1926/1965}
}

@classictext{josephus:ant:thackery,
  author = {Josephus},
  title = {Ant\adddot},
  translator = {Thackeray},
  series = {Loeb Classical Library},
  shortseries = {LCL},
  xref = {josephus}
}
\end{lstlisting}

\begin{biboutput}
  \sampleparencite[2.233-235]{josephus:ant:thackery}
  \samplecite{5}[2.233-235]{josephus:ant:thackery}
  \samplebib{josephus}
  \samplebiblist{josephus}
\end{biboutput}

\begin{refimp}
  (Josephus, \emph{Ant.}\ 2.233–235 [Thackeray, LCL])

  \hspace*{\bibindent}5. Josephus, \emph{Ant.}\ 2.233–235 (Thackeray, LCL).

  \hangindent\bibindent \emph{Josephus.} Translated by Henry St.\@ J.
  Thackeray et al. 10 vols. LCL. Cambridge: Harvard University Press,
  1926–1965.

  \refbiblist{LCL}{Loeb Classical Library}
\end{refimp}

\medskip

\begin{lstlisting}
@mvbook{tacitus,
  author = {Tacitus},
  title = {The Histories and The Annals},
  translator = {Clifford H. Moore and John Jackson},
  volumes = {4},
  series = {Loeb Classical Library},
  shortseries = {LCL},
  location = {Cambridge},
  publisher = {Harvard University Press},
  date = {1937}
}

@classictext{tacitus:ann:jackson,
  author = {Tacitus},
  title = {Ann\adddot},
  translator = {Jackson},
  series = {Loeb Classical Library},
  shortseries = {LCL},
  xref = {tacitus}
}
\end{lstlisting}

\begin{biboutput}
  \samplecite{6}[15.18-19]{tacitus:ann:jackson}
  \samplebib{tacitus}
  \samplebiblist{tacitus}
\end{biboutput}

\begin{refimp}
  \hspace*{\bibindent}6. Tacitus, \emph{Ann.}\ 15.18–19 (Jackson, LCL).

  \hangindent\bibindent Tacitus. \emph{The Histories and The Annals.}
  Translated by Clifford H. Moore and John Jackson. 4 vols. LCL. Cambridge:
  Harvard University Press, 1937.

  \refbiblist{LCL}{Loeb Classical Library}
\end{refimp}

\medskip

\begin{lstlisting}
@mvbook{josephus,
  title = {Josephus},
  translator = {Thackeray, Henry St.\@ J. and others},
  volumes = {10},
  series = {Loeb Classical Library},
  shortseries = {LCL},
  location = {Cambridge},
  publisher = {Harvard University Press},
  date = {1926/1965}
}

@book{josephus:ant1-19,
  author = {Josephus, Flavius},
  shortauthor = {Josephus},
  title = {The Jewish Antiquities, Books 1--19},
  shorttitle = {Ant\adddot},
  translator =  {Thackeray, Henry St.\@ J. and others},
  series = {Loeb Classical Library},
  shortseries = {LCL},
  location = {Cambridge},
  publisher = {Harvard University Press},
  date = {1930/1965},
  xref = {josephus},
  options = {skipbib}
}
\end{lstlisting}

\begin{biboutput}
  \samplecite{14}{josephus:ant1-19}
  \samplebib{josephus}
  \samplebiblist{josephus}
\end{biboutput}

\begin{refimp}
  \hspace*{\bibindent}14. Flavius Josephus, \emph{The Jewish Antiquities,
  Books 1–19,} trans.\@ Henry St.\@ J. Thackeray et al., LCL (Cambridge:
  Harvard University Press, 1930–1965).

  \hangindent\bibindent \emph{Josephus.} Translated by Henry St.\@ J. Thackeray
  et al. 10 vols. LCL. Cambridge: Harvard University Press, 1926–1965.

  \refbiblist{LCL}{Loeb Classical Library}
\end{refimp}

\subsubsection{Papyri, Ostraca, and Epigraphica}

\paragraph{Papyri and Ostraca in General}

\begin{lstlisting}
@ancienttext{p.cair.zen.,
  editor = {Edgar, C. C.},
  title = {Zenon Papyri, Catalogue général des antiquités égyptiennes du Musée du Caire},
  location = {Cairo},
  shorthand = {P.Cair.Zen.},
  options = {skipbib,useeditor=false}
}
\end{lstlisting}

\begin{biboutput}
  \sampleparencite[59003]{p.cair.zen.}
  \samplecite{22}[59003]{p.cair.zen.}
  \samplebiblist{p.cair.zen.}
\end{biboutput}

\begin{refimp}
  (P.Cair.Zen.\ 59003)

  \hspace*{\bibindent}22. P.Cair.Zen.\ 59003.
\end{refimp}

\medskip

\begin{lstlisting}
@book{hunt+edgar:1932,
  author = {Hunt, Arthur S. and Edgar, Campbell C.},
  title = {Select Papyri},
  series = {Loeb Classical Library},
  shortseries = {LCL},
  location = {Cambridge},
  publisher = {Harvard University Press},
  date = {1932}
}

@ancienttext{p.cair.zen.:hunt+edgar,
  editor = {Edgar, C. C.},
  title = {Zenon Papyri, Catalogue général des antiquités égyptiennes du Musée du Caire},
  location = {Cairo},
  shorthand = {P.Cair.Zen.},
  options = {skipbib,useeditor=false},
  related = {hunt+edgar:1932},
  relatedoptions = {useshorttitle=false}
}
\end{lstlisting}

\begin{biboutput}
  \samplecite{22}[(59003)1:96]{p.cair.zen.:hunt+edgar}
  \samplecite{22}[(59003)§31]{p.cair.zen.:hunt+edgar}
  \samplebib{hunt+edgar:1932}
  \samplebiblist{hunt+edgar:1932}
  \samplebiblist*{p.cair.zen.:hunt+edgar}
\end{biboutput}

\begin{refimp}
  \hspace*{\bibindent}22. P.Cair.Zen.\ 59003 (Arthur S. Hunt and Campbell C.
  Edgar, \emph{Select Papyri,} LCL [Cambridge: Harvard University Press,
  1932], 1:96).

  \hspace*{\bibindent}22. P.Cair.Zen. 59003 (Hunt and Edgar §31).

  \refbiblist{LCL}{Loeb Classical Library}
\end{refimp}

\paragraph{Epigraphica}

\paragraph{Greek Magical Papyri}

\begin{lstlisting}
@ancienttext{PGM,
  editor = {Preisendaz, Karl},
  title = {Papyri Graecae Magicae: Die griechischen Zauberpapyri},
  shorttitle = {PGM},
  edition = {2},
  location = {Stuttgart},
  publisher = {Teubner},
  date = {1973/1974},
  shorthand = {PGM},
  options = {skipbib}
}
\end{lstlisting}

\begin{biboutput}
  \sampleparencite[III. 1-164]{PGM}
  \samplecite{22}[III. 1-164]{PGM}
  \samplebiblist{PGM}
\end{biboutput}

\begin{refimp}
  (\emph{PGM} III. 1–164)

  \hspace*{\bibindent}22. \emph{PGM} III. 1–164.

  \refbiblist{\emph{PGM}}{\emph{Papyri Graecae Magicae: Die griechischen
    Zauberpapyri.} Edited by Karl Preisendanz. 2nd~ed. Stuttgart: Teubner,
  1973–1974}
\end{refimp}

\medskip

\begin{lstlisting}
@book{betz:1996,
  author = {Betz, Hans Dieter},
  title = {The Greek Magical Papyri in Translation, Including the Demotic Spells},
  edition = {2},
  location = {Chicago},
  publisher = {University of Chicago Press},
  date = {1996}
}

@ancienttext{PGM:betz,
  editor = {Preisendaz, Karl},
  title = {Papyri Graecae Magicae: Die griechischen Zauberpapyri},
  shorttitle = {PGM},
  edition = {2},
  location = {Stuttgart},
  publisher = {Teubner},
  date = {1973/1974},
  shorthand = {PGM},
  related = {betz:1996},
  relatedoptions = {usefullcite=false,useshorttitle=false}
}
\end{lstlisting}

\begin{biboutput}
  \samplecite{22}[(III. 1-164)]{PGM:betz}
  \samplebib{betz:1996}
  \samplebiblist{PGM:betz}
\end{biboutput}

\begin{refimp}
  \hspace*{\bibindent}22. \emph{PGM} III. 1–164 (Betz).

  \hangindent\bibindent Betz, Hans Dieter. \emph{The Greek Magical Papyri in
  Translation, Including the Demotic Spells.} 2nd~ed. Chicago: University of
  Chicago Press, 1996.

  \refbiblist{\emph{PGM}}{\emph{Papyri Graecae Magicae: Die griechischen
    Zauberpapyri.} Edited by Karl Preisendanz. 2nd~ed. Stuttgart: Teubner,
  1973–1974}
\end{refimp}

\subsubsection{Ancient Epistles and Homilies}

\begin{lstlisting}
@classictext{heraclitus:epistle1,
  author = {Heraclitus},
  title = {Epistle 1}
}
\end{lstlisting}

\begin{biboutput}
  \sampleparencite[10]{heraclitus:epistle1}
  \samplecite{34}[10]{heraclitus:epistle1}
\end{biboutput}

\begin{refimp}
  (Heraclitus, \emph{Epistle 1,} 10)

  \hspace*{\bibindent}34. Heraclitus, \emph{Epistle 1,} 10.
\end{refimp}

\begin{lstlisting}
@book{malherbe:1977,
  editor = {Malherbe, Abraham J.},
  title = {The Cynic Epistles: A Study Edition},
  series = {Stuttgarter Bibelstudien},
  shortseries = {SBS},
  number = {12},
  location = {Atlanta},
  publisher = {Scholars Press},
  date = {1977}
}

@classictext{heraclitus:epistle1:worley,
  author = {Heraclitus},
  title = {Epistle 1},
  translator = {Worley, David},
  pages = {187},
  crossref = {malherbe:1977}
}
\end{lstlisting}

\begin{biboutput}
  \samplecite{36}[10]{heraclitus:epistle1:worley}
  \samplebib{heraclitus:epistle1:worley}
  If \texttt{malherbe:1977} is referenced \texttt{microssrefs} times or more
  then:\par
  \samplebib{heraclitus:epistle1:worley}
  \samplebib*{malherbe:1977}
  \samplebiblist{malherbe:1977}
\end{biboutput}

\begin{refimp}
  \hspace*{\bibindent}36. Heraclitus, \emph{Epistle 1,} 10 (Worley).

  \hangindent\bibindent Heraclitus. \emph{Epistle 1.} Translated by David
  Worley. Page 187 in \emph{The Cynic Epistles: A Study Edition.} Edited by
  Abraham J. Malherbe. SBS~12. Atlanta: Scholars Press, 1977.

  \hangindent\bibindent Malherbe, Abraham J., ed. \emph{The Cynic Epistles: A
  Study Edition.} SBS~12. Atlanta: Scholars Press, 1977.

  \refbiblist{SBS}{Stuttgarter Bibelstudien}
\end{refimp}

\subsubsection{\emph{ANF} and \emph{NPNF}, First and Second Series}

\begin{lstlisting}
@mvcollection{ANF,
  crossref = {ANF:abbreviation},
  editor = {Roberts, Alexander and Donaldson, James},
  origdate = {1885/1887},
  volumes = {10},
  location = {Peabody, MA},
  publisher = {Hendrickson},
  date = {1994},
  shorthand = {ANF},
  options = {useeditor=false,skipbiblist},
  sortkey = {Ante-Nicene Fathers},
  xref = {ANF:abbreviation}
}

@mvcollection{ANF:abbreviation,
  title = {The Ante-Nicene Fathers},
  shorttitle = {ANF},
  shorthand = {ANF},
  options = {skipbib}
}

@classictext{clementinehomilies,
  entrysubtype = {churchfather},
  title = {The Clementine Homilies},
  related = {ANF},
  relatedoptions = {useeditor=false,skipbiblist}
}
\end{lstlisting}

\begin{biboutput}
  \samplecite{14}[(1.3)8:223]{clementinehomilies}
  \samplebib{ANF}
  \samplebiblist{ANF:abbreviation}
\end{biboutput}

\begin{refimp}
  \hspace*{\bibindent}14. \emph{The Clementine Homilies} 1.3 (\emph{ANF}
  8:223).

  \hangindent\bibindent \emph{The Ante-Nicene Fathers.} Edited by Alexander
  Roberts and James Donaldson. 1885–1887. 10 vols.\footnote{Should be “10
  vols.\ 1885–1887.”?} Repr., Peabody, MA:
  Hendrickson, 1994.

  \refbiblist{\emph{ANF}}{\emph{Ante-Nicene Fathers}}
\end{refimp}

\medskip

\begin{lstlisting}
@mvcollection{NPNF1,
  crossref = {NPNF1:abbreviation},
  editor = {Schaff, Philip},
  origdate = {1886/1889},
  volumes = {14},
  location = {Peabody, MA},
  publisher = {Hendrickson},
  date = {1994},
  shorthand = {NPNF\textsuperscript{1}},
  options = {useeditor=false,skipbiblist},
  sortkey = {Nicene and Post-Nicene Fathers, Series 1},
  xref = {NPNF1:abbreviation}
}

@mvcollection{NPNF1:abbreviation,
  title = {The Nicene and Post-Nicene Fathers, \mkbibemph{Series~1}},
  shorttitle = {NPNF\textsuperscript{1}},
  shorthand = {NPNF\textsuperscript{1}},
  options = {skipbib}
}

@classictext{augustine:letters,
  entrysubtype = {churchfather},
  author = {Augustine},
  title = {The Letters of St.\@ Augustin},
  shorttitle = {Letters of St.\@ Augustin},
  volume = {1},
  crossref = {NPNF1},
  related = {NPNF1},
  relatedoptions = {useeditor=false,skipbiblist,skipbib},
  options = {skipbib=false}
}
\end{lstlisting}

\begin{biboutput}
  \samplecite{44}[(28.3.5)252]{augustine:letters}
  \samplebib{augustine:letters}
  \samplebiblist{NPNF1:abbreviation}
\end{biboutput}

\begin{refimp}
  \hspace*{\bibindent}44. Augustine, \emph{Letters of St.\@ Augustin} 28.3.5
  (\emph{NPNF\textsuperscript{1}} 1:252).
  
  \hangindent\bibindent Augustine. \emph{The Letters of St.\@ Augustin.} In
  vol.~1 of \emph{The Nicene and Post-Nicene Fathers,} Series~1. Edited by
  Philip Schaff. 1886–1889. 14 vols.\footnote{Should be “14 vols.\
  1886–1889.”?} Repr., Peabody, MA: Hendrickson, 1994.

  \refbiblist{\emph{NPNF\textsuperscript{1}}}{\emph{The Nicene and Post-Nicene
  Fathers,} Series 1}
\end{refimp}

\subsubsection{J.-P. Migne's Patrologia Latina and Patrologia Graeca}

\begin{lstlisting}
@series{PL,
  title = {Patrologia Latina},
  editor = {Migne, J.-P.},
  volumes = {217},
  location = {Paris},
  date = {1844/1864},
  shorthand = {PL}
}

@series{PG,
  title = {Patrologia Graeca},
  editor = {Migne, J.-P.},
  volumes = {162},
  location = {Paris},
  date = {1857/1886},
  shorthand = {PG}
}

@classictext{gregory:orationestheologicae,
  entrysubtype = {churchfather},
  author = {{Gregory of Nazianzus}},
  title = {Orationes theologicae},
  volume = {36},
  related = {PG},
  relatedoptions = {useeditor=false}
}
\end{lstlisting}

\begin{biboutput}
  \samplecite{6}[(4)12c]{gregory:orationestheologicae}
  \samplebib{PG}
  \samplebib*{PL}
  \samplebiblist{PG}
  \samplebiblist*{PL}
\end{biboutput}

\begin{refimp}
  \hspace*{\bibindent}6. Gregory of Nazianzus, \emph{Orationes theologicae} 4
  (PG 36:12c).

  \hangindent\bibindent Patrologia Graeca. Edited by J.-P. Migne. 162 vols.
  Paris, 1857–1886.

  \hangindent\bibindent Patrologia Latina. Edited by J.-P. Migne. 217 vols.
  Paris, 1844–1864.

  \refbiblist{PG}{Patrologia Graeca. Edited by Jacques-Paul Migne. 162 vols.
  Paris, 1857–1886}

  \refbiblist{PL}{Patrologia Latina. Edited by Jacques-Paul Migne. 217 vols.
  Paris, 1844–1864}
\end{refimp}

\subsubsection{Strack-Billerbeck, \emph{Kommentar zum Neuen Testament}}

\begin{lstlisting}
@mvbook{Str-B,
  author = {Strack, Hermann L. and Billerbeck, Paul},
  title = {Kommentar zum Neuen Testament aus Talmud und Midrasch},
  volumes = {6},
  location = {Munich},
  publisher = {Beck},
  date = {1922/1961},
  shorthand = {Str-B}
}
\end{lstlisting}

\begin{biboutput}
  \samplemacro{\textbackslash autocite[See the discussion of \textbackslash
  textgreek\{ἐκρατοῦντο\} in][2:271]\{Str-B\}}
  \samplecite*{3}[See the discussion of \textgreek{ἐκρατοῦντο} in][2:271]{Str-B}
  \samplebib{Str-B}
  \samplebiblist{Str-B}
\end{biboutput}

\begin{refimp}
  \hspace*{\bibindent}3. See the discussion of \textgreek{ἐκρατοῦντο} in Str-B
  2:271.

  \hangindent\bibindent Strack, Hermann L., and Paul Billerbeck.
  \emph{Kommentar zum Neuen Testament aus Talmud und Midrasch.} 6 vols.
  Munich: Beck, 1922–1961.

  \sloppy\refbiblist{Str-B}{Strack, Hermann Leberecht\footnote{Should be
    “Strack, Hermann L.,”?} and Paul Billerbeck. \emph{Kommentar zum Neuen
    Testament aus Talmud und Midrasch.} 6 vols. Munich: Beck, 1922–1961}
\end{refimp}

\subsubsection{\emph{Aufstieg und Niedergang der römischen Welt (ANRW)}}

\begin{lstlisting}
@mvcollection{ANRW,
  editor = {Temporini, Hildegard and Haase, Wolfgang},
  title = {Aufstieg und Niedergang der römischen Welt: Geschichte und Kultur Roms im Spiegel der neueren Forschung.},
  shorttitle = {ANRW},
  titleaddon = {Part 2, \mkbibemph{Principat}},
  location = {Berlin},
  publisher = {de Gruyter},
  date = {1972/},
  shorthand = {ANRW}
}

@ancienttext{anderson:pepaideumenos,
  entrysubtype = {ANRW},
  author = {Anderson, Graham},
  title = {The \mkbibemph{pepaideumenos} in Action: Sophists and Their Outlook in the Early Empire},
  shorttitle = {\mkbibemph{Pepaideumenos}},
  volume = {33},
  part = {1},
  pages = {80-208},
  location = {New York},
  publisher = {de Gruyter},
  date = {1989},
  related = {ANRW},
  relatedoptions = {skipbib},
  options = {skipbib=false}
}
\end{lstlisting}

\begin{biboutput}
  \samplecite{76}{anderson:pepaideumenos}
  \samplecite{79}[86]{anderson:pepaideumenos}
  \samplebib{anderson:pepaideumenos}
  \samplebib*{ANRW}
  \samplebiblist{ANRW}
\end{biboutput}

\begin{refimp}
  \hspace*{\bibindent}76. Graham Anderson, “The \emph{pepaideumenos} in
  Action: Sophists and Their Outlook in the Early Empire,” \emph{ANRW}
  33.1:80–208.

  \hspace*{\bibindent}79. Anderson, “\emph{Pepaideumenos,}” \emph{ANRW}
  33.1:86.

  \hangindent\bibindent Anderson, Graham. “The \emph{pepaideumenos} in Action:
  Sophists and Their Outlook in the Early Empire.” \emph{ANRW} 33.1:80–208.
  Part 2, \emph{Principat,} 33.1. Edited by H. Temporini and W. Haase. New
  York: de Gruyter, 1989.

  \hangindent\bibindent Temporini, Hildegard, and Wolfgang Haase, eds.
  \emph{Aufstieg und Niedergang der romischen Welt: Geschichte und Kultur Roms
  im Spiegel der neueren Forschung.} Part 2, \emph{Principat.} Berlin: de
  Gruyter, 1972–.

  \refbiblist{\emph{ANRW}}{\emph{Aufstieg und Niedergang der römischen Welt:
    Geschichte und Kultur Roms im Spiegel der neueren Forschung.} Part 2,
    \emph{Principat.} Edited by Hildegard Temporini and Wolfgang Haase.
  Berlin: de Gruyter, 1972–}
\end{refimp}

\subsubsection{Bible Commentaries}

\begin{lstlisting}
@commentary{hooker:1991,
  author = {Hooker, Morna},
  title = {The Gospel according to Saint Mark},
  series = {Black's New Testament Commentaries},
  shortseries = {BNTC},
  number = {2},
  location = {Peabody, MA},
  publisher = {Hendrickson},
  date = {1991}
}
\end{lstlisting}  

\begin{biboutput}
  \samplecite{6}[223]{hooker:1991}
  \samplebib{hooker:1991}
  \samplebiblist{hooker:1991}
\end{biboutput}

\begin{refimp}
  \hspace*{\bibindent}8. Morna Hooker, \emph{The Gospel according to Saint
  Mark,} BNTC~2 (Peabody, MA: Hendrickson, 1991), 223.

  \hangindent\bibindent Hooker, Morna. \emph{The Gospel according to Saint
  Mark.} BNTC~2. Peabody, MA: Hendrickson, 1991.

  \refbiblist{BNTC}{Black’s New Testament Commentaries}
\end{refimp}

\paragraph{Articles and Notes in Study Bibles}

\begin{lstlisting}
@inbook{petersen:2006,
  author = {Petersen, David L.},
  title = {Ezekiel},
  pages = {1096-1167},
  booktitle = {The HarperCollins Study Bible Fully Revised and Updated: New Revised Standard Version, with the Apocryphal\slash Deuterocanonical Books},
  editor = {Attridge, Harold W. and others},
  location = {San Francisco},
  publisher = {HarperSanFrancisco},
  date = {2006}
}
\end{lstlisting}

\begin{biboutput}
  \samplecite{3}[1096]{petersen:2006}
  \samplecite{5}[1096]{petersen:2006}
  \samplebib{petersen:2006}
\end{biboutput}

\begin{refimp}
  \hspace*{\bibindent}3. David L. Petersen, “Ezekiel,” in \emph{The
  HarperCollins Study Bible Fully Revised and Updated: New Revised Standard
  Version, with the Apocryphal\slash Deuterocanonical Books,} ed.\@ Harold W.
  Attridge et al. (San Francisco: HarperSanFrancisco, 2006),
  1096.\footnote{Should be “1096–167.”?}

  \hspace*{\bibindent}5. Petersen, “Ezekiel,” 1096.

  \hangindent\bibindent Petersen, David L. “Ezekiel.” Pages
  1096–1167\footnote{Should be “1096–167”} in \emph{The HarperCollins Study
  Bible Fully Revised and Updated, New Revised Standard Version, with the
  Apocryphal\slash Deuterocanonical Books.} Edited by Harold W. Attridge et
  al. San Francisco: HarperSanFrancisco, 2006.
\end{refimp}

\paragraph{Single-Volume Commentaries on the Entire Bible}

\begin{lstlisting}
@incommentary{partain:1995,
  author = {Partain, Jack G.},
  title = {Numbers},
  pages = {175-179},
  booktitle = {Mercer Commentary on the Bible},
  editor = {Mills, Watson E. and others},
  location = {Macon, GA},
  publisher = {Mercer University Press},
  date = {1995}
}
\end{lstlisting}

\begin{biboutput}
  \samplecite{5}{partain:1995}
  \samplecite{8}[175]{partain:1995}
  \samplebib{partain:1995}
\end{biboutput}

\begin{refimp}
  \hspace*{\bibindent}5. Jack G. Partain, “Numbers,” in \emph{Mercer
  Commentary on the Bible,} ed.\@ Watson E. Mills et al. (Macon, GA: Mercer
  University Press, 1995), 175–79.

  \hspace*{\bibindent}8. Partain, “Numbers,” 175.

  \hangindent\bibindent Partain, Jack G. “Numbers.” Pages 175–79 in
  \emph{Mercer Commentary on the Bible.} Edited by Watson E. Mills et al.
  Macon, GA: Mercer University Press, 1995.
\end{refimp}

\subsubsection{Multivolume Commentaries}

\paragraph{Multivolume Commentaries on a Single Biblical Book by One Author}

\begin{lstlisting}
@mvcommentary{dahood:1965-1970,
  author = {Dahood, Mitchell},
  title = {Psalms},
  volumes = {3},
  series = {Anchor Bible},
  shortseries = {AB},
  number = {16--17A},
  location = {Garden City, NY},
  publisher = {Doubleday},
  date = {1965/1970}
}
\end{lstlisting}

\begin{biboutput}
  \samplecite{4}[3:127]{dahood:1965-1970}
  \samplecite{7}[2:121]{dahood:1965-1970}
  \samplebib{dahood:1965-1970}
  \samplebiblist{dahood:1965-1970}
\end{biboutput}

\begin{refimp}
  \hspace*{\bibindent}4. Mitchell Dahood, \emph{Psalms,} 3 vols., AB 16–17A
  (Garden City, N.Y.:\footnote{Should be “NY:”?} Doubleday, 1965–1970), 3:127.

  \hspace*{\bibindent}7. Dahood, \emph{Psalms,} 2:121.

  \hangindent\bibindent Dahood, Mitchell. \emph{Psalms.} 3 vols. AB 16–17A.
  Garden City, N.Y.:\footnote{Should be “NY:”?} Doubleday, 1965–1970.

  \refbiblist{AB}{Anchor Bible}
\end{refimp}

\medskip

\begin{lstlisting}
@commentary{dahood:1965,
  author = {Dahood, Mitchell},
  title = {Psalms I: 1--50},
  shorttitle = {Psalms I: 1--50},
  volume = {1},
  maintitle = {Psalms},
  series = {Anchor Bible},
  shortseries = {AB},
  number = {16},
  location = {Garden City, NY},
  publisher = {Doubleday},
  date = {1965}
}

@commentary{dahood:1968,
  author = {Dahood, Mitchell},
  title = {Psalms II: 51--100},
  shorttitle = {Psalms II: 51--100},
  volume = {2},
  maintitle = {Psalms},
  series = {Anchor Bible},
  shortseries = {AB},
  number = {17},
  location = {Garden City, NY},
  publisher = {Doubleday},
  date = {1968}
}
\end{lstlisting}

\begin{biboutput}
  \samplecite{78}[44]{dahood:1965}
  \samplecite{79}[78]{dahood:1965}
  \samplecite{82}[374]{dahood:1968}
  \samplebib{dahood:1965}
  \renewbibmacro*{dashcheck}[2]{%
    \usebibmacro{cbx:dashcheck}{#1}{#1}}
  \samplebib*{dahood:1968}
  \samplebiblist{dahood:1965}
\end{biboutput}

\begin{refimp}
  \hspace*{\bibindent}78. Mitchell Dahood, \emph{Psalms I: 1–50,} AB 16
  (Garden City, NY: Doubleday, 1965), 44.

  \hspace*{\bibindent}79. Dahood, \emph{Psalms I: 1–50,} 78.

  \hspace*{\bibindent}82. Mitchell Dahood, \emph{Psalms II: 51–100,} AB 17
  (Garden City, NY: Doubleday, 1968), 347.

  \hspace*{\bibindent}86. Dahood, \emph{Psalms II: 51–100,} 351.

  \hangindent\bibindent Dahood, Mitchell. \emph{Psalms I: 1–50.} Vol.~1 of
  \emph{Psalms.} AB~16. Garden City, NY: Doubleday, 1965.

  \hangindent\bibindent\refbibnamedash \emph{Psalms II: 51–100.} Vol.~2 of
  \emph{Psalms.} AB~17. Garden City, N.Y.:\footnote{Should be “NY:”?}
  Doubleday, 1968.

  \refbiblist{AB}{Anchor Bible}
\end{refimp}

\paragraph{Multivolume Commentaries for the Entire Bible by Multiple Authors}

\begin{lstlisting}
@mvcommentary{NIB,
  editor = {Keck, Leander E.},
  title = {The New Interpreter's Bible},
  shorttitle = {NIB},
  volumes = {12},
  location = {Nashville},
  publisher = {Abingdon},
  date = {1994/2004},
  shorthand = {NIB}
}

@incommentary{miller:2001,
  author = {Miller, Patrick D.},
  title = {The Book of Jeremiah: Introduction, Commentary, and Reflections},
  pages = {553-926},
  booktitle = {Introduction to Prophetic Literature, Isaiah, Jeremiah, Baruch, Letter of Jeremiah, Lamentations, Ezekiel},
  volume = {6},
  date = {2001},
  crossref = {NIB},
  xref = {NIB}
}
\end{lstlisting}

\begin{biboutput}
  \samplecite{1}[577]{miller:2001}
  \samplebib{miller:2001}
  \samplemacro{\textbackslash usepackage[style=sbl,fullbibrefs]\{biblatex\} \\
  \textbackslash printbibliography}
  \toggletrue{fullbibrefs}
  \samplebib*{miller:2001}
  \togglefalse{fullbibrefs}
  \samplebiblist{NIB}
\end{biboutput}

\begin{refimp}
  \hspace*{\bibindent}1. Patrick D. Miller, \emph{NIB} 6:577.

  \hangindent\bibindent Miller, Patrick D. “The Book of Jeremiah:
  Introduction, Commentary, and Reflections.” Pages 553–926 in
  \emph{Introduction to Prophetic Literature, Isaiah, Jeremiah, Baruch, Letter
  of Jeremiah, Lamentations, Ezekiel.} Vol.~6 of \emph{New Interpreter’s
  Bible.} Edited by Leander E. Keck.\footnote{Should include “12 vols.”?}
  Nashville: Abingdon, 2001.

  \hangindent\bibindent Miller, Patrick D. “The Book of Jeremiah:
  Introduction, Commentary, and Reflections.” \emph{NIB} 6:553–926.

  \refbiblist{\emph{NIB}}{\emph{The New Interpreter’s Bible.} Edited by
  Leander E. Keck. 12 vols. Nashville: Abingdon, 1994–2004}
\end{refimp}

\subsubsection{SBL Seminar Papers}

\begin{lstlisting}
@seminarpaper{crenshaw:2001,
  author = {Crenshaw, James L.},
  title = {Theodicy in the Book of the Twelve},
  booktitle = {Society of Biblical Literature 2001 Seminar Papers},
  series = {Society of Biblical Literature Seminar Papers},
  shortseries = {SBLSP},
  number = {40},
  location = {Atlanta},
  publisher = {Society of Biblical Literature},
  date = {2001},
  pages = {1-18}
}
\end{lstlisting}

\begin{biboutput}
  \samplecite{33}{crenshaw:2001}
  \samplebib{crenshaw:2001}
  \samplebiblist{crenshaw:2001}
\end{biboutput}

\begin{refimp}
  \hspace*{\bibindent}33. James L. Crenshaw, “Theodicy in the Book of the
  Twelve,” \emph{Society of Biblical Literature 2001 Seminar Papers,}
  SBLSPS\footnote{Should be “SBLSP”?}~40 (Atlanta: Society of Biblical
  Literature, 2001), 1–18.
  
  \hangindent\bibindent Crenshaw, James L. “Theodicy in the Book of the
  Twelve.” Pages 1–18 in \emph{Society of Biblical Literature 2001 Seminar
  Papers.} SBLSPS\footnote{Should be “SBLSP”?}~40. Atlanta: Society of
  Biblical Literature, 2001.

  \refbiblist{SBLSP}{Society of Biblical Literature Seminar Papers}
\end{refimp}

\subsubsection{A CD-ROM Reference (with a Corresponding Print Edition)}

Books on CD-ROM should be cited according to the print edition. It is not
necessary to indicate the medium in the citation.

\subsubsection{Text Editions Published Online with No Print Counterpart}

\begin{lstlisting}
@online{wilhelm:2013,
  editor = {Wilhelm, Gernot},
  title = {Der Vertrag Šuppiluliumas I. von Ḫatti mit Šattiwazza von Mitrani \mkbibparens{CTH 51.I}},
  shorttitle = {Der Vertrag Šuppiluliumas I},
  eprintdate = {2013-02-24},
  eprint = {CTH 51.I},
  eprintclass = {INTR 2013-02-24},
  eprinttype = {hethiter}
}
\end{lstlisting}

\begin{biboutput}
  \samplecite{2}{wilhelm:2013}
  \samplecite{4}{wilhelm:2013}
  \samplebib{wilhelm:2013}
\end{biboutput}

\begin{refimp}
  \hspace*{\bibindent}2. Gernot Wilhelm, ed., “Der Vertrag Šuppiluliumas I.
  von Ḫatti mit Šattiwazza von Mitrani (CTH 51.I),” released 24 February 2013,
  doi:hethiter.net/: CTH 51.I (INTR 2013-02-24).

  \hspace*{\bibindent}4. Wilhelm, “Der Vertrag Šuppiluliumas I.”

  \hangindent\bibindent Gernot Wilhelm, ed. “Der Vertrag Šuppiluliumas I. von
  Ḫatti mit Šattiwazza von Mitrani (CTH 51.I).”\footnote{Should this include
  “Released 24 February 2013.”?} doi:hethiter.net/: CTH 51.I
  (INTR 2013-02-24).
\end{refimp}

\subsubsection{Online Database}

\begin{lstlisting}
@online{cobb:figurines,
  author = {{Cobb Institute of Archaeology}},
  title = {The Figurines of Maresha, the Persian Era},
  eprint = {http://www.cobb.msstate.edu/dignew/Maresha/index.html},
  eprinttype = {DigMaster},
  options = {indexing=false}
}
\end{lstlisting}

\begin{biboutput}
  \samplecite{37}{cobb:figurines}
  \samplebib{cobb:figurines}
\end{biboutput}

\begin{refimp}
  \hspace*{\bibindent}37. Cobb Institute of Archaeology. “The Figurines of
  Maresha, the Persian Era,” DigMaster,
  \nolinkurl{http://www.cobb.msstate.edu/dignew/Maresha/index.html}.

  \hangindent\bibindent Cobb Institute of Archaeology. “The Figurines of
  Maresha, the Persian Era.” DigMaster.
  \nolinkurl{http://www.cobb.msstate.edu/dignew/Maresha/index.html}.
\end{refimp}

\begin{lstlisting}
@online{caraher:2013,
  editor = {Caraher, William R.},
  title = {Pyla-Koutsopetria Archaeological Project: \mkbibparens{Overview}},
  eprint = {http://opencontext.org/projects/3F6DCD13-A476-488E-ED10-47D25513FCB2},
  eprinttype = {Open Context},
  doi = {10.6078/M7B56GNS},
  eprintdate = {2013-11-05}
}
\end{lstlisting}

\begin{biboutput}
  \samplecite{15}{caraher:2013}
  \samplecite{17}{caraher:2013}
  \samplebib{caraher:2013}
\end{biboutput}

\begin{refimp}
  \hspace*{\bibindent}15. William R. Caraher, ed., “Pyla-Koutsopetria
  Archaeological Project: (Overview),” Open Context, released 5 November 2013,
  \nolinkurl{http://opencontext.org/projects/3F6DCD13-A476-488E-ED10-47D25513FCB2},
  \nolinkurl{doi:10.6078/M7B56GNS}.

  \hspace*{\bibindent}17. Caraher, “Pyla-Koutsopetria Archaeological Project.”

  \hangindent\bibindent William R. Caraher, ed.\footnote{Should be “Caraher,
  William R., ed.?”} “Pyla-Koutsopetria Archaeological Project: (Overview).”
  Open Context. Released 5 November 2013.
  \nolinkurl{http://opencontext.org/projects/3F6DCD13-A476-488E-ED10-47D25513FCB2}.
  \nolinkurl{doi:10.6078/M7B56GNS}.\sloppy
\end{refimp}

\subsubsection{Websites and Blogs}

\begin{lstlisting}
@online{100cuneiform,
  title = {The One Hundred Most Important Cuneiform Objects},
  eprint = {http://cdli.ox.ac.uk/wiki/doku.php?id=the_one_hundred_most_ important_cuneiform_objects},
  eprinttype = {cdli:wiki}
}
\end{lstlisting}

\begin{biboutput}
  \samplecite{10}{100cuneiform}
  \samplebib{100cuneiform}
\end{biboutput}

\begin{refimp}
  \hspace*{\bibindent}10. “The One Hundred Most Important Cuneiform Objects,”
  cdli:wiki,
  \nolinkurl{http://cdli.ox.ac.uk/wiki/doku.php?id=the_one_hundred_most_important_cuneiform_objects}.

  \hangindent\bibindent “The One Hundred Most Important Cuneiform Objects.”
  cdli:wiki.
  \nolinkurl{http://cdli.ox.ac.uk/wiki/doku.php?id=the_one_hundred_most_important_cuneiform_objects}.
\end{refimp}

\begin{lstlisting}
@online{goodacre:2014,
  author = {Goodacre, Mark},
  title = {Jesus' Wife Fragment: Another Round-Up},
  journaltitle = {NT Blog},
  date = {2014-05-09},
  url = {http://ntweblog.blogspot.com}
}
\end{lstlisting}

\begin{biboutput}
  \samplecite{3}{goodacre:2014}
\end{biboutput}

\begin{refimp}
  \hspace*{\bibindent}3. Mark Goodacre, “Jesus’ Wife Fragment: Another
  Round-Up,” \emph{NT Blog,} 9 May 2014,
  \nolinkurl{http://ntweblog.blogspot.com}.
\end{refimp}

\section{Other Examples}

\subsection{BDAG, BDB, BDF}

\begin{lstlisting}
@lexicon{BDAG,
  author = {Danker, Frederick W. and Bauer, Walter and Arndt, William F. and Gingrich, F. Wilbur},
  title = {Greek-English Lexicon of the New Testament and Other Early Christian Literature.},
  edition = {3},
  location = {Chicago},
  publisher = {University of Chicago Press},
  date = {2000},
  shorthand = {BDAG}
}
\end{lstlisting}

\begin{biboutput}
  \samplecite{1}[35]{BDAG}
  \samplebib{BDAG}
  \samplebiblist{BDAG}
\end{biboutput}

\begin{refimp}
  \refbiblist{BDAG}{Danker, Frederick W., Walter Bauer, William F. Arndt, and
    F. Wilbur Gingrich. \emph{Greek-English Lexicon of the New Testament and
    Other Early Christian Literature.} 3rd~ed. Chicago: University of Chicago
  Press, 2000}
\end{refimp}

\medskip

\begin{lstlisting}
@lexicon{BDB:abbreviation,
  author = {Brown, Francis and Driver, S. R. and Briggs, Charles A.},
  title = {A Hebrew and English Lexicon of the Old Testament},
  shorthand = {BDB},
  options = {skipbib}
}

@lexicon{BDB,
  crossref = {BDB:abbreviation},
  location = {Oxford},
  publisher = {Oxford University Press},
  date = {1906},
  shorthand = {BDB},
  xref = {BDB:abbreviation},
  options = {skipbiblist}
}
\end{lstlisting}

\begin{biboutput}
  \samplecite{2}[432]{BDB}
  \samplebib{BDB}
  \samplebiblist{BDB:abbreviation}
\end{biboutput}

\begin{refimp}
  \refbiblist{BDB}{Brown, Francis, S. R. Driver, and Charles A. Briggs.
  \emph{A Hebrew and Aramaic Lexicon of the Old Testament}}
\end{refimp}

\medskip

\begin{lstlisting}
@book{BDF,
  author = {Blass, Friedrich and Debrunner, Albert and Funk, Robert W.},
  title = {A Greek Grammar of the New Testament and Other Early Christian Literature},
  location = {Chicago},
  publisher = {University of Chicago Press},
  date = {1961},
  shorthand = {BDF}
}
\end{lstlisting}

\begin{biboutput}
  \samplecite{3}[§441]{BDF}
  \samplebib{BDF}
  \samplebiblist{BDF}
\end{biboutput}

\begin{refimp}
  \refbiblist{BDF}{Blass, Friedrich, Albert Debrunner, and Robert W. Funk.
    \emph{A Greek Grammar of the New Testament and Other Early Christian
  Literature.} Chicago: University of Chicago Press, 1961}
\end{refimp}

\subsection{\emph{HALOT, TLOT}}

\begin{lstlisting}
@mvlexicon{HALOT,
  author = {Koehler, Ludwig and Baumgartner, Walter and Stamm, Johann J.},
  title = {The Hebrew and Aramaic Lexicon of the Old Testament},
  shorttitle = {HALOT},
  editor = {Richardson, Mervyn E. J.},
  editortype = {translated and edited under the supervision of},
  volumes = {4},
  location = {Leiden},
  publisher = {Brill},
  date = {1994/1999},
  shorthand = {HALOT}
}
\end{lstlisting}

\begin{biboutput}
  \samplecite{4}[2:223]{HALOT}
  \samplebib{HALOT}
  \samplebiblist{HALOT}
\end{biboutput}

\begin{refimp}
  \refbiblist{\emph{HALOT}}{\emph{The Hebrew and Aramaic Lexicon of the Old
    Testament.} Ludwig Koehler, Walter Baumgartner, and Johann J. Stamm.
    Translated and edited under the supervision of Mervyn E. J. Richardson. 4
  vols. Leiden: Brill, 1994–1999}
\end{refimp}

\medskip

\begin{lstlisting}
@mvlexicon{TLOT,
  title = {Theological Lexicon of the Old Testament},
  shorttitle = {TLOT},
  editor = {Jenni, Ernst},
  witheditor = {Westermann, Claus},
  witheditortype = {withassistance},
  translator = {Biddle, Mark E.},
  volumes = {3},
  location = {Peabody, MA},
  publisher = {Hendrickson},
  date = {1997},
  shorthand = {TLOT}
}
\end{lstlisting}

\begin{biboutput}
  \samplecite{5}[1:24]{TLOT}
  \samplebib{TLOT}
  \samplebiblist{TLOT}
\end{biboutput}

\begin{refimp}
  \refbiblist{\emph{TLOT}}{\emph{Theological Lexicon of the Old Testament.}
    Edited by Ernst Jenni, with assistance from Claus Westermann. Translated
  by Mark E. Biddle. 3~vols. Peabody, MA: Hendrickson, 1997}
\end{refimp}

\subsection{\emph{SBLHS}}

\begin{lstlisting}
@manual{SBLHS,
  title = {SBL Handbook of Style: For Biblical Studies and Related Disciplines},
  shorttitle = {SBLHS},
  edition = {2},
  location = {Atlanta, GA},
  publisher = {SBL Press},
  date = {2014},
  shorthand = {SBLHS}
}
\end{lstlisting}

\begin{biboutput}
  \samplecite{1}[§6.2.1]{SBLHS}
  \samplebib{SBLHS}
  \samplebiblist{SBLHS}
\end{biboutput}

\begin{refimp}
  \refbiblist{\emph{SBLHS}}{\emph{Society of Biblical Literature Handbook of
  Style.} 2nd~ed. Atlanta, GA: SBL Press, 2014}
\end{refimp}

\clearpage

\phantomsection
\printbiblist[heading=biblistintoc]{abbreviations}

\clearpage

\phantomsection
\printbibliography[heading=bibintoc,notcategory=ignore]

\clearpage

\rmfamily
\printindex

\end{document}
